% Marco Teorico.
\chapter{Reconocimiento de locutor} \label{chap:Reconocimiento de locutor}

\section{Introducci\'on}

La forma mas natural que tenemos para comunicarnos es hablando, ya que de esta manera es como transmitimos nuestras ideas cotidianamente. Con los avances tecnol\'ogicos y conforme los dispositivos electr\'onicos se integran mas en nuestra vida y en la sociedad, cobra mayor relevancia acceder a estos sistemas a trav\'es del habla.\\

El objetivo de la comunicaci\'on es transmitir informaci\'on, la señal de voz logra transmitir informaci\'on en muchos niveles, en un nivel primario, un porcentaje de la informaci\'on es intencional, es decir el interlocutor activamente articula un mensaje concreto a trav\'es de la concatenaci\'on de fonemas de un lenguaje espec\'ifico. Sin embargo en un nivel secundario, otro porcentaje de la informaci\'on transmitida, es propia del interlocutor, como lo es el sexo, edad, estado de \'animo y de salud, incluso referencias geogr\'aficas, esta informaci\'on tambi\'en se transmite a trav\'es del habla de manera natural, y en la mayor\'ia de los casos el interlocutor no realiza ningun esfuerzo adicional o consciente para transmitirla.\\

El proceso por el cu\'al se transmite la informaci\'on, comienza con la representaci\'on del mensaje de forma abstracta en la mente del interlocutor, esta se convierte en señales neuronales que controlan los mecanismos articulatorios (lengua, labios, cuerdas vocales, etc.), estos se mueven en respuesta a dichas señales, produciendo una secuencia de gestos, que finalmente producen una señal ac\'ustica en la cual se encuentra el mensaje codificado, m\'as otra informaci\'on que como ya habiamos mencionado, es propia del locutor.\\

Una forma de caracterizar la comunicaci\'on oral, es en el contexto de la teor\'ia de la informaci\'on, esta nos dice que el habla puede representarse en t\'erminos del contenido del mensaje. Por ejemplo, se estima que los humanos podemos producir en promedio 10 fonemas por segundo \cite{rabiner1987}, tomando en cuenta que cada lenguaje tiene un conjunto de entre 30 y 50 fonemas, podemos representar cada fonema con un n\'umero binario de seis bits, esto nos da un estimado de la velocidad en la que se transmite la informaci\'on de 60 bits por segundo e promedio.\\

La representaci\'on de la señal del habla debe ser tal que sea f\'acil para un humano o una m\'aquina poder extraer la informaci\'on, as\'i que en realidad una buena representaci\'on de la señal (que no tome en cuenta solamente la transmisi\'on del mensaje) en el contexto del reconocimiento de locutor, puede llegar a requerir entre 500 y un mill\'on de bits por segundo \cite{rabiner1987}. Aunque la caracterizaci\'on que nos brinda la teor\'ia de la informaci\'on es clara y cuantificable, en la pr\'actica es mas \'util apoyarse en una caracterizaci\'on de la onda ac\'ustica, ya sea en el dominio del tiempo, o en el de las frecuencias.\\

Como se ha mencionado anteriormente, la señal de voz en su representaci\'on ac\'ustica ha resultado muy \'util en aplicaciones pr\'acticas, ya que transmite informaci\'on multidimensional, esta propiedad hace posible que se pueda extraer informaci\'on variada, en particular hay tres tipos principales de informaci\'on que es posible extraer: el mensaje hablado, el lenguaje y la identidad del locutor. De esta distinci\'on se generan tres tipos de sistemas de reconocimiento: el reconocimiento de voz, reconocimiento de lenguaje y el reconocimiento de locutor. De estos tres el mas popular es el reconocimiento de voz, sin embargo el reconocimiento de locutor va cobrando mayor relevancia y cada vez recibe mas atenci\'on. Esta tesis se centra exclusivamente en el reconocimiento de locutor.\\



\section{Conceptos fundamentales}

El reconocimiento de locutor es el proceso mediante el cual se extrae, caracteriza y clasifica la informaci\'on de la señal de voz con la finalidad de reconocer la identidad de la persona que produce dicha señal. En algunos contextos y dependiendo de la finalidad del sistema se le denota como biometr\'ia de voz, esto es por que se usan propiedades fisiol\'ogicas del individuo (en este caso el tracto vocal) para reconocer su identidad, y aunque en algunos casos es comprensible que se defina de esta manera, en la mayor\'ia de casos es preferible el t\'ermino de reconocimiento de locutor, por ser mas general. Cabe destacar el valor del reconocimiento de locutor como biometr\'ia, por sus posibilidades de realizarse a distancia a trav\'es de una linea telef\'onica, esto le da mayor flexibilidad y relevancia.\\

En la literatura se manejan varias divisiones para el reconocimiento de locutor, en \cite{beigi2011} el autor maneja seis posibles ramas que a su vez divide en dos grupos: simples y compuestas. En el primer grupo incluye: la identificaci\'on, la verificaci\'on y la clasificaci\'on, mientras que en el segundo grupo incluye: la segmentaci\'on, la detecci\'on y el seguimiento. A las ramas del primer grupo las considera simples por ser autocontenidas y realizar una funci\'on espec\'ifica, mientras que al segundo grupo las considera compuestas por apoyarse en una o m\'as manifestaciones de las ramas simples con ayuda de t\'ecnicas extra.\\ 

Otros autores \cite{reynolds2002} manejan tres ramas: identificaci\'on, clasificaci\'on y verificaci\'on las cuales corresponden al grupo que se denota como simple en \cite{beigi2011}. En la mayor parte de la literatura solo se manejan dos distintas ramas: la identificaci\'on de locutor y la verificaci\'on de locutor, en esta tesis tomaremos este \'ultimo enfoque, por considerarlo el mas pr\'actico, por tener el grado de especificidad suficiente para el problema que se aborda y por que las ramas que quedan fuera pueden ser consideradas como subramas de la identificaci\'on de locutor. \\

Actualmente la verificaci\'on de locutor, es la rama mas popular del reconocimiento de locutor, debido a su importancia en aplicaciones de seguridad y en control de accesos, cabe mencionar que la implementaci\'on de estos sistemas presenta menos dificultades que las que puede presentar un sistema de identificaci\'on de locutor.

\subsection{Verificaci\'on de locutor}
La verificaci\'on de locutor (o tambien denominada autenticaci\'on de locutor o biometr\'ia de voz) consiste en que un individuo, llamado locutor de prueba, se identifica mediante alg\'un mecanismo alternativo a la voz, esto puede ser un id de registro, un nombre de usuario, una contraseña, etc. Este ID esta asociado a un modelo ac\'ustico de locutor en la base de datos, a este modelo se le denomina modelo de locutor objetivo, el locutor de prueba ingresa su señal de voz y esta es comparada con el modelo objetivo, con la finalidad de verificar si la señal del locutor de prueba puede pertenecer al locutor objetivo, y de ser asi se realiza la verificaci\'on de manera positiva, de lo contrario no se confirma la verificaci\'on del locutor de prueba.\\

Debido a la compleja naturaleza de la señal de voz, resulta impr\'actico hacer una comparaci\'on exclusivamente entre el locutor de prueba y el locutor objetivo, ya que se puede tener una medici\'on del parecido que guardan ambas señales, pero no se tiene una referencia del nivel de parecido que ser\'a aceptable para el sistema. Para resolver esto se suele introducir un tercer modelo que ayuda a contrastar la similitud entre el modelo de prueba y el modelo objetivo, dando una referencia del nivel que similitud que pueda ser significativo en la verificaci\'on. Este modelo de referencia debe tener ciertas propiedades que logren generalizar las caracter\'isticas si no de toda la poblaci\'on, al menos de un grupo representativo de esta. Existen varios enfoques para contrastar el nivel de similitud introduciendo un tercer modelo, a continuaci\'on se mencionan los dos enfoques principales que se manejan en la bibliogra\'ia.\\

Uno de los m\'etodos mas populares es el  UBM (Universal Background Model por sus siglas en ingl\'es) este modelo se contruye con datos de una gran parte de la poblaci\'on, de esta manera se mide la similitud que el locutor de prueba guarda con el locutor objetivo y se contrasta con la similitud que guarda con el resto de la poblaci\'on, de tal manera que si la primera es mayor que la segunda, se confirma la verificaci\'on, de lo contrario se rechaza.\\

Otro m\'etodo muy popular es el modelo de cohortes, en este m\'etodo, se contruye el modelo de contraste con  individuos de la poblaci\'on que guardan similitud con el locutor objetivo, se realiza un contraste similar al del caso anterior para confirmar la verificaci\'on. Este metodo a pesar de que utiliza una poblaci\'on menor para el modelo de contraste, se puede decir que es mas robusto, ya que el modelo de contraste sera muy parecido al modelo del locutor objetivo.\\

En ambos casos, como se puede apreciar, para realizar la verificaci\'on de locutor, solo se necesita realizar dos comparaciones, esto los convierte en un sistema de f\'acil escalamiento, ya que el computo necesario para su realizaci\'on se mantiene constante independientemente del n\'umero de usuarios registrados.\\

\subsection{Identificaci\'on de locutor}
La identificaci\'on de locutor es el proceso por el cu\'al se encuentra la identidad de un locutor desconocido. A diferencia de la verificaci\'on de locutor en que se contrasta el registro de entrada solamente con dos modelos (el modelo del usuario que se afirma ser y el modelo de contraste) el proceso de identificaci\'on de locutor conlleva una comparaci\'on de uno a muchos, esto por que el modelo de entrada se contrasta con cada uno de los modelos registrados en el sistema, se calcula el nivel de similitud y se reporta el modelo que obtenga la similitud mas alta.\\

Existen dos tipos de sistemas de indentificaci\'on de locutor, los sistemas de conjunto cerrado y los sistemas de conjunto abierto. Los sistemas de conjunto cerrado son los mas sencillos, en estos sistemas el audio del locutor de prueba es contrastado con cada uno de los locutores registrados, y el sistema asigna la identidad del locutor con el que se registra la mayor similitud, cabe mencionar que es posible que el locutor de prueba no se encuentre en el conjunto de los usuarios registrados, a\'un as\'i el sistema devuelve un ID de alguno de los usuarios registrados, en algunos contextos esto puede que no sea lo ideal, los sistemas de conjunto abierto resuelven este problema.\\ 

Los sistemas de conjunto abierto pueden verse como una combinaci\'on de un sistema de identificaci\'on de locutor de conjunto cerrado y un sistema de verificaci\'on de locutor, ya que primero se hace una comparaci\'on uno a muchos, se elige el perfil con el que se obtenga la mayor similitud y posteriormente se realiza una verificaci\'on. Si la verificaci\'on se realiza de manera positiva se se acepta la identidad encontrada por la identificaci\'on de conjunto cerrado, en cambio si la verificaci\'on es rechazada, el sistema no acepta ning\'un ID como valido y se considera que el locutor de prueba no se encuentra registrado en el sistema.\\

Un sistema de verificaci\'on es an\'alogo a un sistema de identificaci\'on de locutor de conjunto abierto en el caso l\'imite en que s\'olo hay un usuario registrado, en el sentido que ambos tienen la misma complejidad. En general un sistema de identificaci\'on de locutor es mas complejo que uno de reconocimiento de locutor, ya que te\'oricamente se compara el locutor de prueba contra todos los modelos de locutores de la base de datos, esto hace que la complejidad crezca de forma lineal conforme el n\'umero de usuarios registrados aumenta.\\

\subsection{Fases del reconocimiento de locutor}

La mayor\'ia de sistemas de aprendizaje supervisado, constan de al menos dos etapas que son fundamentales para su funcionamiento, la etapa de entrenamiento que es necesaria antes de poder hacer un uso efectivo del modelo y la etapa de generalizaci\'on, en donde se usa el modelo para realizar la tarea para la cu\'al fu\'e entrenado. De igual manera los sistemas de reconocimiento de locutor se componen de dos fases, inicialmente es necesaria una fase de registro o enrolamiento de locutores, es an\'aloga a la etapa de entrenamiento, para que el sistema pueda realizar el reconocimiento, necesita saber que es lo que tiene que reconocer, posteriormente viene la fase de reconocimiento o verificaci\'on de locutores, es la tarea para la cu\'al fu\'e diseñado el sistema.\\

En la fase de registro dependiendo del tipo de reconocimiento de locutor y la modalidad, los usuarios ingresan alguna frase o se usa alguna grabaci\'on de su voz y se asigna un c\'odigo \'unico de identificaci\'on de usuario, tambi\'en llamado ID, de esta grabaci\'on se extraen caracter\'isticas ac\'usticas que seran utilizadas para entrenar el algoritmo y generar un modelo de voz del usuario, finalmente, este modelo es almacenado en la base de datos mapeado a la identidad del usuario.\\

En la fase de reconocimiento o verificaci\'on, se toma la señal de voz de un usuario de identidad desconocida, y se aplica la misma t\'ecnica de extracci\'on de caractr\'isticas ac\'usticas que ha sido empleada en la fase de registro, posteriormente estas caracter\'isticas se comparan con los usuarios registrados en la base de datos y se mide la similitud que guarda con cada uno de los modelos registrados y dependiendo de esta se realiza el reconocimiento.\\
 
\subsection{Modalidades del reconocimiento de locutor}
El reconocimiento de locutor puede ser implementado de distintas maneras, cada una de estas formas puede aportar alg\'un beneficio a cambio de imponer requerimientos extra, estas modalidades dependen del contexto, de la flexibilidad en la captaci\'on de la señal de voz, del tipo de reconocimiento que se vaya a implementar y de las necesidades del sistema. La verificaci\'on de locutor es la rama que mas se beneficia, por ser la que se da en un entorno mas controlado, y debido a que algunos de los requerimientos se pueden implementar de forma natural en un sistema de verificaci\'on. Con la finalidad de mejorar la calidad en el reconocimiento o reforzar la seguridad del sistema, estas modalidades exploran formas de incorporar otras fuentes de informaci\'on en conjunto a la señal ac\'ustica.\\ 

\textbf{\large Reconocimiento de locutor independiente de texto}\\

El reconocimiento de locutor independiente de texto es la modalidad mas versatil, esta modalidad que puede ser usada en todas las ramas del reconocimiento de locutor. Un sistema de reconocimiento de locutor independiente de texto, toma en cuenta \'unicamente las caracter\'isticas vocales de cada usuario, y dependiendo de su configuraci\'on no se apoya en ning\'un tipo de informaci\'on extra, en algunos casos incluso el lenguaje es de poca relevancia para el sistema.\\ 

En estos sistemas el texto que se usa en la etapa de registro difiere del texto con el que se compara durante la etapa de verificaci\'on, es por esto que los sistemas de identificaci\'on de locutor son independientes del texto, ya que requieren poca coperaci\'on por parte del usuario, incluso se puede hacer un enrolamiento sin el conocimiento del usuario, esto es de gran utilidad en aplicaciones forenses.\\

Por otro lado, en estos sistemas se requiere un tiempo de registro mas amplio, ya que es necesario cubrir el mayor n\'umero posible de fonemas con la finalidad de aumentar la probabilidad de tener la referencia en la etapa de reconocimiento. Otro problema que se puede encontrar en estos sistemas es para sistemas de verificaci\'on el hecho de poder ingresar cualquier sucesi\'on de palabras puede presentar deficiencias en seguridad.\\  

\textbf{\large Reconocimiento de locutor dependiente de texto}\\

En un sistema dependiente de texto, el texto en la fase de verificaci\'on es el mismo que el utilizado en la fase de registro, o el sistema espera recibir un texto predertiminado en la fase de verificaci\'on. Existen varias formas en que se determina el texto que se solicita. El texto puede ser com\'un para todos los usuarios, es decir, que a los usuarios se les solicite ingresar una frase com\'un. Otra forma es que el usuario elija o se le asigne una contraseña, lo cu\'al puede reforzar la seguridad del sistema. Tambi\'en hay sistemas basados en conocimiento, que explotan informaci\'on del usuario como podr\'ia ser el nombre, su direcci\'on o n\'umero telef\'onico. Finalmente existen sistemas que solicitan al usuario que pronuncia alguna frase de verificaci\'on que puede haber sido o no registrada en la fase de enrrolamiento, estos sistemas pueden estar reforzados por un sistema de reconocimiento de voz.\\

Los sistemas de reconocimiento de locutor dependientes de texto son implementados exclusivamente para sistemas de verificaci\'on de locutor, por la naturaleza de estos sitemas es f\'acil solicitar a los usuarios que registren alguna frase predeterminada, ya que la fase de registro se da en un ambiente controlado y el usuario se registra voluntariamente. En cambio, para sistemas de identificaci\'on de locutor resulta impr\'actico e incluso imposible solicitar texto a los usuarios, ya que se dan en un entorno de habla mas natural y en la mayoria de los casos los usuarios no tienen conocimiento de que estan registrados en el sistema.\\ 

\section{Sistemas de reconocimiento de locutor}

\subsection{Arquitectura}
Un sistema de reconocimiento de locutor (figura 2.1) consta de varios m\'odulos que se agrupan en tres componentes b\'asicos: procesamiento de la señal para la extracci\'on de caracter\'isticas, el modelo de locutores y el reconocimiento de patrones. El procesamiento de la señal consiste en captar la señal de voz, digitalizarla, y procesar la señal para reducir el ruido y remover silencios, posteriormente se realiza un procesamiento mas complejo el cual tiene como finalidad extraer caracter\'isticas ac\'usticas propias de cada locutor. El modelo de locutores se construye con la informaci\'on que se extrae de la señal, puede ser un modelo estad\'istico o un modelo construido por un algoritmo de aprendizaje de m\'aquina, en el reconocimiento de patrones se realiza una comparaci\'on de patrones entre los modelos construidos anteriormente y una nueva señal,  que permita al sistema tomar una decisi\'on sobre la identidad del locutor.\\ 

\begin{figure}[H]
	\begin{center}
	\includegraphics[scale=0.45,type=png,ext=.png,read=.png]{imagenes/diagrama1} \\
	\caption{Diagrama de un sistema gen\'erico de reconocimiento de locutor.}
	\label{fig:diag_recon_locutor}
	\end{center}
\end{figure}

Estos m\'odulos pueden tener distintas configuraciones dependiendo del diseño del sistema, como se ha mencionado anteriormente el reconocimiento de locutor se divide en dos ramas, identificaci\'on de locutor y verificaci\'on de locutor. La diferencia mas notoria entre ambos sistemas se da en la etapa de comparaci\'on de patrones, ya que para el caso de la verificaci\'on se realizan solamente dos comparaciones, mientras que para el caso de la identificaci\'on se compara con todos los modelos de locutor registrados en la base de datos.\\

En la figura 2.2 se muestra la arquitectura de un sistema de verificaci\'on de locutor, en estos sistemas el usuario ingresa la señal de voz y un ID, el sistema accede a la base de datos para obtener el modelo de locutor registrado con el ID, procesa la señal de voz, extrae las caracter\'isticas ac\'usticas y las compara con el modelo del ID y con un modelo universal o de cohortes, con esta informaci\'on decide si la señal de voz pertenece al ID ingresado y acepta o rechaza la verificaci\'on.\\

\begin{figure}[H]
	\begin{center}
	\includegraphics[scale=0.45,type=png,ext=.png,read=.png]{imagenes/diagrama2} \\
	\caption{Diagrama de un sistema de verificaci\'on de locutor.}
	\label{fig:diag_verif_locutor}
	\end{center}
\end{figure}

En la figura 2.3 se muestra la arquitectura de un sistema de identificaci\'on de locutor, aqu\'i el usuario ingresa la señal de voz, se extraen las caracter\'isticas y se comparan con todos los modelos de la base de datos, y dependiendo si el sistema es de conjunto cerrado o abierto, elige el ID  del usuario con mayor similitud o concluye que la señal de voz no pertenece a ning\'un usuario registrado.\\

\begin{figure}[H]
	\begin{center}
	\includegraphics[scale=0.45,type=png,ext=.png,read=.png]{imagenes/diagrama3} \\
	\caption{Diagrama de un sistema de identificaci\'on de locutor.}
	\label{fig:diag_clasif_locutor}
	\end{center}
\end{figure}

A continuaci\'on se expone la fase de registro y las caracter\'isticas de los tres m\'odulos principales.\\


\subsection{Registro}
La primer fase en un sistema de reconocimiento de locutor es la de registro, esta fase se compone de tres etapas, en la primera se obtienen grabaciones de voz del usuario que ser\'a registrado en el sistema, posteriormente se extraen caracter\'isticas de la señal de voz para tener una representac\i\'on mas adecuada, y finalmente se construye un modelo para cada usuario que ser\'a almacenado en la base de datos. El objetivo de esta fase es el de obtener suficiente informaci\'on de cada usuario que le permita al sistema construir un modelo para realizar la tarea de reconocimiento con un bajo de nivel de error y a su vez, no sea necesaria demasiada informaci\'on resultando en una cantidad impr\'actica para su captura.\\

Esta fase funciona de forma distinta dependiendo del sistema, teniendo distintas caracter\'isticas para los sistemas de verificaci\'on y los de identificaci\'on, de la misma manera pose ventajas y retos propios para cada modalidad ya sea independientes o dependientes de texto.\\

En un sistema independiente de texto, existe la ventaja de que no se impone ninguna restricci\'on al usuario e incluso se puede captar la señal de su voz sin que el usuario este consciente de esto, esto a su vez representa un reto, ya que puede darse el caso extremo de que el registro no cuente con una variedad de fonemas suficientes para lograr el reconocimiento, lo que puede resultar en niveles altos de error o para compensar esto puede ser necesario que los tiempos de registro sean mayores. A pesar de esto, cabe recordar, como ya se hab\'ia mencionado anteriormente, los sistemas independientes de texto son la \'unica modalidad que resulta viable para todos los tipos de reconocimiento de locutor.\\

En un sistema dependiente de texto, la misma frase que se usa en la fase de registro, ser\'a usada para realizar la verificaci\'on, esto lo convierte en una modalidad de las mas simples, ya que no hay tomar en cuenta muchas consideraciones, la frase con la que se registra el usuario puede ser elegida por el mismo, o alguna frase que le solicite el sistema, la frase que sea utilizada es indiferente para la tarea de verificaci\'on ya que ser\'a usada la misma frase posteriormente, y la elecci\'on de esta depender\'a de otros factores.\\

El sistema de texto solicitado, es un sistema dependiente de texto que resulta mas complejo que el anterior, en esta modalidad el sistema le solicita al usuario que ingrese una frase espec\'ifica en la fase de registro que no es necesariamente la misma frase que solicitar\'a el sistema en la fase de verificaci\'on. Una forma de implementar este tipo de sistemas es anticipar todos los posibles fonemas que se usaran en la fase de verificaci\'on y cubrirlos en la fase de registro. Una forma mas com\'un, seria solicitando frases personalizadas para cada usuario, en donde se solicite informaci\'on espec\'ifica de cada usuario, como informaci\'on personal, o asignarle un conjunto de palabras claves a cada usuario.\\

\subsection{Procesamiento de la señal de voz}

El procesamiento de la señal de voz es uno de los pasos mas importantes en cualquier sistema de reconocimiento de locutor,  esta etapa tiene como finalidad extraer la informaci\'on deseada de la señal de voz. El proceso comienza captando la señal ac\'ustica generalmente con un micr\'ofono, convirtiendola en una señal el\'ectrica, este proceso requiere la aplicaci\'on de un filtro para evitar el efecto de alising, esto se logra cortando todas las frecuencias que se encuentren por encima de un medio de la frecuencia de sampleo tambi\'en llamada frecuencia de Nyquist.\\

Una vez que se tiene filtrada la señal anal\'ogica, se digitaliza usando un convertidor A/D (anal\'ogico/digital), los convertidores A/D actuales pueden captar la señal a frecuencias de sampleo de 96,000 muestras por segundo, con una resoluci\'on de hasta 24 bits, aunque com\'unmente se suele utilizar una frecuencia de sampleo de 44,100 muestras por segundo con resoluci\'on de 16 bits. Esto quiere decir que por cada muestra tendremos 16 bits, y por cada segundo tendremos 44,100 muestras, lo que da un total de 705,600 bits por segundo, esta es una cantidad de informaci\'on muy grande considerando que podemos producir 10 fonemas por segundo.\\

Esta señal digitalizada en la forma en que se encuentra resulta poco \'util para la tarea de reconocimiento de locutor, en primer lugar la cantidad de informaci\'on resulta impr\'actica ya que hay mucha informaci\'on inecesaria y redundante, un ejemplo son los fragmentos de silencio y la frecuencia de sampleo, la frecuencia de sampleo de 44,100 Hz capta frecuencias hasta 22,050 Hz, aqu\'i hay informaci\'on  que resulta inutil ya que la voz humana se extiende de los 50 Hz a los 8,000 Hz, por lo tanto la señal puede reducirse eliminando los fragmentos de silencio y cambiando la frecuencia de sampleo a 16,000 muestras por segundo, reduciendo un poco la cantidad de informaci\'on.\\

Lo anterior a pesar de ser necesario para tener una señal mas limpia y manejable, no es suficiente, la señal sigue teniendo un flujo de informaci\'on muy grande, adem\'as, la forma en que esta representada la señal, no es la mas \'optima, la señal se encuentra representada como variaci\'on de presi\'on de aire en el tiempo, esto no es muy \'util para el reconocimiento de locutor, ya que lo que diferencia los distintos tonos de voz de las personas, no es la forma en que modifican la presi\'on de aire, lo que realmente diferencia los tonos de voz, es el contenido espectral. Es por esto que se necesita una representaci\'on de la señal en donde se enfatize el espectro arm\'onico.\\

\begin{figure}[H]
	\begin{center}
	\includegraphics[scale=0.5,type=png,ext=.png,read=.png]{imagenes/diagrama4} \\
	\caption{Captura y procesamiento de la señal.}
	\label{fig:diag_verif_locutor}
	\end{center}
\end{figure}

Las t\'ecnicas m\'as comunes que se utilizan en el reconocimiento de locutor hacen \'enfasis en las cualidades arm\'onicas de la señal, en el caso de el banco de filtros de frecuencias Mel y los coeficientes cepstrales de frecuencias Mel, se aplica la transformada r\'apida de Fourier, que es la version computacional de la transformada de Fourier, lo que hace este algoritmo es que transforma la señal del \'ambito del tiempo al \'ambito de las frecuencias, posteriormente aplica un conjunto de filtros que buscan emular la forma en que el oido humano capta el sonido, estos son los procesos m\'as importantes del c\'alculo, en el cap\'itulo siguiente se presenta todo el procedimiento.\\

Otras dos t\'ecnicas tambi\'en comunes son el c\'odigo de predicci\'on lineal y la predicci\'on de percepci\'on lineal, estas t\'ecnicas tambi\'en hacen \'enfasis en el material arm\'onico, aunque sin el uso de la transformada de Fourier, en estas representaciones se modela el tracto vocal encontrando las resonancias o formantes de la señal, lo que resulta en una representaci\'on de la envolvente del espectro arm\'onico, en el caso de la predicci\'on de percepci\'on lineal a la envolvente se le aplica un banco de filtros que emulan la escucha humana, estas dos t\'ecnicas tambien se desarrollan mas a detalle en el siguiente cap\'itulo.\\

Estas t\'ecnicas no solo enfatizan el espectro arm\'onico de la señal, si no que reducen considerablemente su dimensi\'on, lo que contribuye a tener una señal con informaci\'on mas compacta, enfocada al problema que se busca resolver y manejable tanto en tiempo de procesamiento como en espacio de almacenamiento.

\subsection{Modelo de locutor}

En la fase de registro despu\'es de realizar la extracci\'on de caracter\'isticas de la señal, se construye un modelo de locutor propio del usuario utilizando el vector de caracter\'isticas, el objetivo de crear un modelo de locutores es el de poder asociar a cada usuario con un identificador que permita al sistema diferenciarlo de los dem\'as usuarios. Se considera que algunos de los atributos deseables que debe poseer son: Bases te\'oricas s\'olidas que puedan dar claridad sobre el comportamiento del modelo, que sea generalizable sobre nuevos datos, es decir que no sobre ajuste los datos de registro y que sea pr\'actico en t\'erminos de tiempo de procesamiento y no requiera demasiado espacio de almacenamiento. Este proceso por estar en el centro de cualquier aplicaci\'on de reconocimiento de locutor puede ser considerado como el m\'as importante, es por esto que debe ser abordado con las consideraciones suficientes en el diseño del sistema.\\

Existen varias t\'ecnicas que han sido empleadas en el reconocimiento de locutor y la elecci\'on de una de estas depende de las caracter\'isticas de la señal, del rendimiento esperado del sistema, la facilidad para entrenarlo y actualizarlo y principalmente de la modalidad del sistema de reconocimiento. A continuaci\'on se enlistan algunas de las t\'ecnicas mas com\'unmente usadas.\\

\textbf{Tiempo din\'amico de deformaci\'on.- } En este modelo se crea una plantilla, que consiste en un conjunto de vectores pertenecientes a alguna frase fija que se obtiene en la etapa de registro, posteriormente en la etapa de reconocimiento la misma frase es ingresada por el usuario, este algoritmo alinea ambas frases para que su longitud sea la misma, este paso es necesario debido a la variabilidad del habla, posteriormente se mide la similitud que guardan entre si, esta t\'ecnica es casi exclusiva para los sistemas de verificaci\'on dependientes de texto, ya que en estos casos su rendimiento es mayor.\\

\textbf{Modelos ocultos de Markov.- } En estos modelos se codifica la evoluci\'on temporal de las caracter\'isticas y se realiza un modelo estad\'istico de su variaci\'on, de esta manera se obtiene una representaci\'on estad\'istica de la manera en que cada usuario produce sonido. Durante la fase de registro se obtienen los par\'ametros del modelo y en la fase de reconocimiento se estima la verosimilitud de que la secuencia ingresada pertenezca al modelo. En sistemas independientes de texto se suele usar un modelo de un solo estado, conocido como modelo de mezcla de Gaussianas (Gaussian Mixture Model), en este modelo se asigna una distribuci\'on normal a cada usuario y se calculan sus par\'ametros en la fase de registro, posteriormente en la fase de reconocimiento para el caso de identificaci\'on de locutor se busca la distribuci\'on en la que la señal guarda la m\'axima verosimilitud.\\

\textbf{Cuantificaci\'on vectorial.- } Esta es una t\'ecnica de clusterizaci\'on en la que se mapean los vectores de un espacio vectorial a un conjunto finito de regiones en dicho espacio, es un algoritmo de compresi\'on de datos con perdida o tambi\'en llamada compresi\'on irreversible. En el contexto del reconocimiento de locutor el algoritmo utiliza los vectores de caracter\'isticas de la fase de registro para generar una partici\'on del mismo tamaño que el n\'umero de usuarios registrados, cada clase es representada por el centroide de la regi\'on a la que pertenece llamado vector c\'odigo, el conjunto de todos los vectores c\'odigo es llamado libro de c\'odigos, el cu\'al sirve como modelo de los locutores, este conjunto es significativamente menor al conjunto de vectores de la fase de registro. En la etapa de reconocimiento se utiliza una m\'etrica que mide la similitud entre los vectores de entrada y cada vector c\'odigo, t\'ipicamente es la medida de distorsi\'on de cuantificaci\'on entre dos conjuntos de vectores, finalmente se asigna a la clase en que la distorsi\'on es m\'inima.\\

\textbf{Redes neuronales artificiales.- } Las redes neuronales artificiales son una familia de modelos que pueden encontrar relaciones complejas en los datos al mismo tiempo que extraen caracter\'isticas, por esta raz\'on puede que para un sistema de reconocimiento de locutor que utilice redes neuronales resulte inecesaria la etapa de extracci\'on de caracter\'isticas. Gracias a las distintas formas de representaci\'on de la señal de audio, entre los distintos modelos de redes neuronales existen varios que se adaptan a las necesidades del reconocimiento de locutores. La señal de audio por ser una serie de tiempo unidimensional, puede ser modelada por redes neuronales recurrentes o por una red de memoria a corto y largo plazo, ambas arquitecturas est\'an diseñadas para procesar y encontrar relaciones en secuencias temporales. Otra arquitectura es la de las redes neuronales convolucionales que est\'a diseñada para procesar im\'agenes y encontrar relaciones espaciales, si la señal de audio se representa con el espectrograma este puede ser procesado por este tipo de red neuronal. Puede haber otras arquitecturas que funcionen para el procesamiento de señales de voz como las redes neuronales profundas, sin embargo se ha demostrado que en general las redes neuronales artificiales son computacionalmente muy costosas y que en algunos casos los modelos no son generalizables \cite{reynolds2002}.\\

 


\subsection{Reconocimiento de patrones}

El reconocimiento de patrones o comparaci\'on de patrones se encuentra en la fase de reconocimiento, es posterior a la fase de registro ya que es necesario que los modelos de locutores est\'en almacenados. Esta tarea consiste en asignar una puntuaci\'on de coincidencia, la cu\'al es una medida de similitud entre un vector de caracter\'isticas que a\'un no ha sido observado por el sistema y puede o no pertenecer a alguno de los usuarios registrados y alg\'un modelo de la base de datos.\\

Esta medida de similitud depende del modelo de locutores que ha sido implementado en el sistema, en \cite{campbell1997} se menciona que existen dos tipos de modelos: los de plantilla y los estoc\'asticos. Para los modelos de plantilla como el tiempo din\'amico de deformaci\'on y la cuntificaci\'on de vectores, se mide de manera determin\'istica, con m\'etricas de distancia como la distancia Euclidea o la distancia de Mahalanobis, mientras que para los modelos estoc\'asticos como los modelos ocultos de Markov, o la mezcla de Gaussianas se mide de manera probabil\'istica, con medidas de verosimilitud o probabilidad condicional.\\

Esta medida es utilizada para la decisi\'on que tomar\'a el sistema en el reconocimiento, esta desici\'on depende del tipo de reconocimiento del sistema, siendo distinta en el caso de la verificaci\'on al de identificaci\'on y tambi\'en en el diseño del sistema.\\ 

Para el caso de la verificaci\'on de locutor, una vez obtenida la metrica de similitud, el sistema necesita una referencia para saber si esta medida es suficiente para aceptar o rechazar la verificaci\'on, esta referencia esta representada en el diagrama de la figura 2.2 como el umbral, este umbral puede obtenerse de tres formas. El caso mas sencillo es en el que durante el diseño del sistema se define un umbral fijo para todos los usuarios, si la similitud esta por debajo de este umbral se rechaza la verificaci\'on y si es mayor se acepta. Otra forma de obtener el umbral, es con la similitud que guarda la señal con un modelo de referencia, en la literatura se manejan dos formas de construir este modelo, el modelo universal de fondo y el modelo de cohortes, estos modelos lo que buscan es representar las caracter\'isticas generales de una poblaci\'on o de un subconjunto espec\'ifico de la poblaci\'on para obtener un contraste significativo de similitud, si la similitud de la señal con el modelo de veirficaci\'on es mayor que la similitud con el modelo de referencia, la verificaci\'on es aceptada, en caso contrario se rechaza. Dependiendo de las caracter\'isticas del sistema y de la poblaci\'on se recomienda uno sobre el otro.\\



En el caso de la identificaci\'on de locutor, se mide la similitud que la señal guarda con cada uno de los modelos registrados y dependiendo si es un sistema de conjunto cerrado, puede elegir alg\'un usuario con el que guarde la similitud mas alta, o si es un sistema de conjunto abierto, elegir al usuario con mayor similitud o descartarlos a todos y concluir que es un usuario desconocido. Estos sistemas tienen la caracter\'istica que a mayor n\'umero de usuarios registrados el proceso de reconocimiento se prolonga.\\




%La principal forma que tenemos los humanos para comunicarnos es hablando, y la función de comunicarnos es la transmisión de información desde una fuente hacia un receptor, la persona que habla es la fuente y el receptor generalmente es otra persona. Esa información suele presentarserse y representarse en distintas formas. Comienza como impulsos eléctricos en el cerebro, se envian señales a los m\'usculos del tracto vocal para que genere y modele una onda acústica, para el caso de las aplicaciones de reconocimiento de habla el receptor es una máquina que recibe la señal acústica por medio de micr\'ofonos que convierten la onda acústica en impulsos eléctricos, estos pasan por un convertidor anal\'ogico/digital que samplea y quantiza la señal eléctrica conviertiendola en datos que la computadora puede leer, generalmente representado como arreglos de números.\\

%Estos arreglos de números son una representación de la onda acústica, entre la información que contiene esta representación se encuentra el mensaje que quiere enviar el interlocutor, su estado emocional, el leguage que usa, el ambiente acústico que lo rodea, caracterí­sticas propias de los micr\'ofonos y filtros utilizados, entre otras. La información que nos concierne es la información que es propia de la persona que esta hablando. Para poder separar de la señal digitalizada la información que nos interesa, se le da un tratamiento a la señal que entre otras cosas, extrae cualidades de percepción humana, reduce la dimensión de los datos, y deshecha información que no nos interesa.



