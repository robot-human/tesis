%% ----------------------------------------------------------------
%% Thesis.tex -- MAIN FILE (the one that you compile with LaTeX)
%% ---------------------------------------------------------------- 

% file thesis.tex
% Archivo thesis.tex
% Documento maestro que incluye todos los paquetes necesarios para el documento
% principal.

% Documento obtenido por un sinfin de iteraciones de administradores del LDC
% Estructura actual hecha por:
% Jairo Lopez <jairo@ldc.usb.ve>
% Actualizado ligeramente por:
% Alexander Tough 
% Actualizado con una estructura de carpetas por:
% Tony Lattke

% Set up the document
\documentclass[oneside,11pt,letterpaper]{report}
\tolerance=1000  
\hbadness=10000  
\raggedbottom

% Para escribir algoritmos
\usepackage{listings}
\usepackage{algpseudocode}
\usepackage{algorithmicx}
\usepackage{algorithm}

\usepackage{lscape}

% Paquetes para manejar graficos
\usepackage{import}
\usepackage{epsf}
\usepackage[pdftex]{graphicx}
\usepackage{epsfig}

% Para items
\usepackage{enumitem}

% Simbolos matematicos
\usepackage{latexsym,amssymb}
%
% Paquetes para presentar una tesis decente.
\usepackage{setspace,cite} % Doble espacio para texto, espacio singular para
                           % los caption y pie de pagina
                           	% los comandos de espacio son: \doublespacing \onehalfspace  \singlespace  \spacing{1.5}
                                                      
\usepackage[table]{xcolor}
\usepackage{tikz}
\usetikzlibrary{shapes.geometric,arrows}

\usepackage{pgfplots}

\usetikzlibrary{arrows,shapes}
\usepackage{verbatim} % Needed for the "comment" environment to make LaTeX comments

\usepackage{comment}

\usepackage{wrapfig}
\usepackage{alltt}

% Acentos 
\usepackage[spanish,activeacute,es-noquoting]{babel}

\usepackage[spanish]{translator}
\usepackage[utf8]{inputenc}
\usepackage{color, xcolor, colortbl}
\usepackage{multirow}
\usepackage{subfig}
\usepackage[OT1]{fontenc}
\usepackage{tocbibind}
\usepackage{anysize}
\usepackage{listings} 

\decimalpoint

% Para poder tener texto asiatico
\usepackage{CJK}

% Opciones para los glosarios
\usepackage[style=altlist,toc,numberline,acronym]{glossaries}
\usepackage{url}
\usepackage{amsthm}
\usepackage{amsmath}
\usepackage{fancyhdr} % Necesario para los encabezados
\usepackage{fancyvrb}
\usepackage{makeidx} % En caso de necesitar indices.
\makeindex  % Necesitado para los indices

% Caleb recomendó que el tamaño del nombre de la imagen fuera más pequeño
\usepackage[font=footnotesize,labelfont=bf]{caption}
% la que le sigue es scriptsize y luego tiny

% Definiciones para definicions, teoremas y lemas
\theoremstyle{definition} \newtheorem{definicion}{Definici\'{o}n}
\theoremstyle{plain} \newtheorem{teorema}{Teorema}
\theoremstyle{plain} \newtheorem{lema}{Lema}

% Para la creacion de los pdfs
\usepackage{hyperref}

\usepackage{upquote}

% Para resolver el lio del Unicode para la informacion de los PDFs
% En pdftitle coloca el nombre de su proyecto de grado/pasantia.
% En pdfauthor coloca su nombre.
\hypersetup{
    pdftitle = {Identificación de locutor para un robot de servicio utilizando aprendizaje profundo},
    pdfauthor= {Pedro Rodriguez Zarazua},
    colorlinks,
    citecolor=black,
    filecolor=black,
    linkcolor=black,
    urlcolor=black,
    backref,
    pdftex
}

\definecolor{brown}{rgb}{0.7,0.2,0}
\definecolor{darkgreen}{rgb}{0,0.6,0.1}
\definecolor{darkgrey}{rgb}{0.4,0.4,0.4}
\definecolor{lightgrey}{rgb}{0.9,0.9,0.9}
\definecolor{softgray}{rgb}{0.7,0.7,0.7}
\definecolor{lightcyan}{rgb}{0.88,1,1}
\definecolor{lightgreen}{RGB}{144,238,144}
\definecolor{khaki}{RGB}{240,230,140}


\usepackage{listings}
\lstnewenvironment{code}{\lstset{basicstyle=\small}}{}

\lstset{escapeinside=~~}
\lstset{
   frame=single,
   framerule=1pt,
   showstringspaces=false,
   basicstyle=\footnotesize\ttfamily,
   keywordstyle=\textbf,
   backgroundcolor=\color{lightgrey}
}

% Para crear la hoja escaneada de las firmas
\usepackage[absolute]{textpos}

% Pone los nombres y las opciones para mostrar los codigos fuentes
\lstset{language=C, breaklines=true, frame=single, showstringspaces=false,
        showtabs=false, numbers=left, keywordstyle=\color{black},
        basicstyle=\footnotesize, captionpos=b }
\renewcommand{\lstlistingname}{C\'{o}digo fuente}
\renewcommand{\lstlistlistingname}{\'{I}ndice de c\'{o}digos fuente}

\newcommand{\todo}{ TODO: }

\theoremstyle{definition}
\newtheorem{definition}{Definición}[section]


% Dimensiones de la pagina
\setlength{\headheight}{15pt}
\marginsize{2cm}{2cm}{2cm}{2cm}


%%%%%%%%%%%%%%%%%%%%%%%%%%%%%%%%%%%%%%%%%%%%%%%%%%%%%%%%%%%%%%%%%%%%%%%%%%%
%%%%%%%%%%%%%%%%      end of preamble and start of document     %%%%%%%%%%%
%%%%%%%%%%%%%%%%%%%%%%%%%%%%%%%%%%%%%%%%%%%%%%%%%%%%%%%%%%%%%%%%%%%%%%%%%%%
\begin{document}

% Pagina de titulo
% Pagina de titulo
\begin{titlepage}
\begin{center}

% Upper part (aqui ya esta incluido el logo de la USB).
%\includegraphics[scale=0.5,type=png,ext=.png,read=.png]{imagenes/cebolla} \\
\includegraphics[scale=0.07,type=png,ext=.png,read=.png]{imagenes/escudo_UNAM} \\

% Encabezado
\textsc {\large UNIVERSIDAD NACIONAL AUT'ONOMA DE M'EXICO} \\
\textsc{\bfseries \\FACULTAD DE CIENCIAS\\
}

\bigskip
\bigskip
\bigskip
\bigskip
\bigskip
\bigskip
\bigskip
%\bigskip
%\bigskip

% Title/Titulo
% Aqui ponga el nombre de su proyecto de grado/pasantia larga
\textsc{\bfseries ESTUDIO COMPARATIVO DEL RENDIMIENTO DE CARACTER\'ISTICAS DE LA SEÑAL DE VOZ PARA EL RECONOCIMIENTO DE LOCUTOR}

\bigskip
\bigskip
\bigskip
\bigskip
\bigskip

% Author and supervisor/Autor y tutor
\begin{minipage}{\textwidth}
\centering
% Bottom half
{TESIS \\ Que para optar por el grado de: \\
Licenciado en Matem\'aticas} \\

\bigskip
\bigskip
\bigskip
\bigskip


Presenta: \\
Pedro Rodriguez Zarazua\\

\bigskip
\bigskip
\bigskip
\bigskip

Director de Tesis: \\
Dr. CALEB ANTONIO RASC'ON ESTEBAN'E
\end{minipage}

\bigskip
\bigskip
\bigskip
\bigskip
\bigskip
\bigskip
\bigskip
\bigskip
\bigskip

\vfill

% Date/Fecha 
{\large \bfseries Ciudad de M'exico, 
%FECHA
Mayo de 2021}

\end{center}
\end{titlepage}


\setcounter{secnumdepth}{3}
\setcounter{tocdepth}{4}

% Define encabezado numeros romanos y como se separan los captiulos y las
% secciones
\addtolength{\headheight}{3pt}
\pagenumbering{roman}
\pagestyle{fancyplain}

\renewcommand{\chaptermark}[1]{\markboth{\chaptername\ \thechapter:\,\ #1}{}}
\renewcommand{\sectionmark}[1]{\markright{\thesection\,\ #1}}

\onehalfspacing

\lhead{}
\chead{}
\rhead{}
\renewcommand{\headrulewidth}{0.0pt}
\lfoot{}
\cfoot{\fancyplain{}{\thepage}}
\rfoot{}

%% Pendiente
% Pagina de resumen
\setcounter{page}{4}
\begin{center}
	{\bf Resumen} \pdfbookmark[0]{Resumen}{resumen} % Sets a PDF bookmark for the dedication
\end{center}	

Durante este trabajo ....

\vspace{15mm}

\textbf{\textit{Palabras clave :}}  Reconocimiento de locutor, Extracci\'on de caracter\'isticas, MFB, MFCC, LPC, PLP, GMM, VQ.


% Pendiente
% Pagina de dedicatoria (opcional)

\setcounter{page}{5}

\chapter*{Dedicatoria
\markboth{Dedicatoria}{Dedicatoria}}
\pdfbookmark[0]{Dedicatoria}{Dedicatoria}

\bigskip

A ...




%% Pendiente
% Pagina de agradecimientos (opcional)
\setcounter{page}{6}

\chapter*{Agradecimientos
\markboth{Agradecimientos}{Agradecimientos}}
\pdfbookmark[0]{Agradecimientos}{agradecimientos}

\bigskip

Agradezco 
%

% Crea la tabla de contenidos
\tableofcontents

% Crea la lista de cuadros
%%%%%%%%%%%%%%%%%%%%%%%%%%%%%%%%%%%%%%%%%%%%%%%%%%%%%%%%%%%%%%%%%
% Renaming the list of tables and tables caption
%%%%%%%%%%%%%%%%%%%%%%%%%%%%%%%%%%%%%%%%%%%%%%%%%%%%%%%%%%%%%%%%%%%
\renewcommand{\listtablename}{Índice de tablas}
\renewcommand\spanishtablename{Tabla}
\listoftables

% Crea la lista de figuras
\listoffigures

%% Crea la lista de codigos fuentes
%\lstlistoflistings

\clearpage

% Define encabezado en numeros arabicos  
\pagenumbering{arabic}

\fancyhf{} % Redefine el encabezado 
\lhead{}
\chead{}
\rhead{\fancyplain{}{\thepage}}
\renewcommand{\headrulewidth}{0.0pt}
\lfoot{}
\cfoot{}
\rfoot{}

% Caleb me recomendó que el espaciado fuera como el del resumen y o doble
%\doublespacing
\onehalfspacing

%\makeglossaries
% Incluye los archivos deseados - El contenido de su proyecto de grado/pasantia larga.
\chapter{Introducci'on} \label{chap:introduccion}

El reconocimiento de locutor es una rama de la inteligencia artificial, que relaciona la psicoac\'ustica, la fisiologia de la voz, el procesamiento de señales digitales y el aprendizaje de m\'aquina. En los problemas relacionados a este campo se busca ya sea identificar, detectar o verificar la identidad de una persona a trav\'es del registro y el an\'alisis de la señal de voz.\\

Esta tarea ha sido de inter\'es en el \'area de las telecomunicaciones desde los años 60 \cite{history2004} teniendo como limitantes la capacidad de almacenamiento y procesamiento de los dispositivos de la \'epoca, sin embargo en las \'ultimas d\'ecadas esto ha dejado de ser un problema y con el desarrollo de la rob\'otica y de las aplicaciones en dispositivos m\'oviles, la necesidad de mejorar la comunicaci\'on hombre-m\'aquina ha hecho que el reconocimiento de voz y el reconocimiento de locutor cobren gran relevancia. \\

El poder reconocer o identificar de forma autom\'atica la identidad de las personas a trav\'es de la señal de voz, podr\'a contribuir entre otras cosas en el desarrollo de sistemas de comunicaci\'on inteligentes capaces de ofrecer experiencias personalizadas. Con esto en mente, en la medida en que los sistemas de reconocimiento de locutor mejoren y sean capaces de extraer de manera autom\'atica la informaci\'on que se encuentra codificada en la señal de voz, se ir\'an refinando, ofreciendo una interacci\'on m\'as natural a los usuarios, ya que tendr\'an mayor flexibilidad y funcionalidades.\\

Algunas aplicaciones del reconocimiento de locutor son: control de accesos, reconocimiento forense, gestión de audio, refuerzos de seguridad, entre otros. Puede llegar a ser muy útil y tener varias aplicaciones prácticas y en general a mejorar la interacci\'on hombre-m\'aquina, por ejemplo, se ha comprobado que dentro de las biometr\'ias es la que tiene mayor aceptación entre  las personas y es la m\'as pr\'actica de realizar a distancia gracias a la telefon\'ia m\'ovil. Tambi\'en resulta de gran ayuda en robots de servicio, robusteciendo la identificaci\'on de personas, ya que en algunas situaciones el reconocimiento facial no es posible, por ejemplo en el caso en que el robot se encuentre en otra habitaci\'on o este realizando una tarea que requiera el uso de la c\'amara o en situaciones donde la iluminaci\'on sea muy d\'ebil, en estos casos, el reconocimiento de locutor ayudar\'a a que el robot sepa quien se esta comunicando con el en todo momento.\\

Los sistemas de reconocimiento de locutor, consisten en registrar la grabaci\'on de voz de un grupo de usuarios conocidos, y extraer de la señal de voz un conjunto de coeficientes que son caracter\'isticos e idealmente \'unicos, propios de la persona que los produce, estos coeficientes se usan para generar un modelo biom\'etrico de cada usuario. Posteriormente se ingresan al sistema otras grabaciones de voz de usuarios de indentidad desconocida, que pueden o no estar registrados, y el sistema realiza el mismo proceso de extracci\'on de coeficientes caracter\'isticos y contrasta el nuevo modelo con los modelos previamente registrados, se mide la relaci\'on que guarda la señal con los modelos registrados y dependiendo del nivel de similitud, el sistema puede realizar alguna afirmaci\'on sobre la identidad del locutor al que pertenece.\\ 

En cualquier sistema de reconocimiento de locutor vamos a encontrar dos m\'odulos que son fundamentales para su funcionamiento, el m\'odulo de extracci\'on de caracter\'isticas y el m\'odulo de comparaci\'on de patrones. En el m\'odulo de extracci\'on de caracter\'isticas se busca extraer de la señal de voz informaci\'on que nos permita crear un modelo \'unico para cada locutor, y posteriormente en el m\'odulo de comparaci\'on de patrones se usan estas caracter\'isticas y se contrastan entre si con la finalidad de identificar si la señal de voz pertenece a alguno de los locutores previamente registrados en el sistema.\\

En el m\'odulo de comparaci\'on de patrones, se utilizan t\'ecnicas de reconocimiento de patrones y aprendizaje de m\'aquina supervisado, de las mas usadas en el reconocimiento de locutor son: Modelos de mezclas de Gaussianas o GMM (Gaussian Mixture Models), los modelos ocultos de Markov o HMM (Hidden Markov Modeling), y la cuatizaci\'on de vectores o VQ (Vector Quantization), estas son las t\'ecnicas que mas aparecen en trabajos previos, como se puede observar en la cronolog\'ia del reconocimiento de locutor que aparece en \cite{campbell1997}.\\

Para el caso del m\'odulo de extracci\'on de caracter\'isticas, algunas de las representaciones de la señal de voz mas utilizadas son, los coeficientes cepstrales de freciencias Mel o MFCC (Mel frequency cepstral coefficients) y los bancos de filtros de escala Mel o MFB (Mel Filter Banks) estos se basan en la forma en que funciona la percepci\'on auditiva humana, los c\'odigos de predicci\'on linear o LPC (Linear predictive coding) que est\'an basados en la forma en que la voz humana es producida, y la predicci\'on linear perceptual o  PLP (Perceptual linear predictive) que son una combinaci\'on de los anteriores. Aunque no son todas las representaciones que existen, son las que se mencionan con mayor frecuencia en la bibliograf\'ia \cite{beigi2011}.\\

Estas representaciones se obtienen a trav\'es de varios pasos que implican distintos c\'alculos y ajustes lo que resulta en una cantidad considerable de par\'ametros a definir, los distintos autores dan recomendaciones de que par\'ametros utilizar en ciertos contextos lo cu\'al resulta \'util en la mayor\'ia de casos, pero como se puede apreciar, las posibles combinaciones de modelos de clasificaci\'on, de representaciones de la señal de voz y de los par\'ametros a definir puede resultar abrumadora a la hora de diseñar un sistema de reconocimiento de locutor.\\ 

En este trabajo se realiza un estudio comparativo de las t\'ecnicas de representaci\'on de la señal de voz MFCC, MFB, LPC y PLP explorando su espacio param\'etrico, utilizando los modelos de reconocimiento de patrones GMM y VQ, evaluando su rendimiento en la tarea de identificaci\'on de locutor para cada caso, con la finalidad de brindar criterios para la elecci\'on de la configuraci\'on en sistemas de reconocimiento de locutor en trabajos futuros.\\




\medskip

\section{Objetivos} \label{sect:objetivos}

\subsection{Objetivo general} \label{subsec:objetivo_general}

El objetivo general de este proyecto es:\\

Dado un sistema de reconocimiento de locutor tener los criterios suficientes para elegir el tipo de caracterí­sticas acústicas que se utilizarán, así­ como el valor de los parámetros que se ajusten mejor a la aplicacion con base en las necesidades del sistema mismo. 
\medskip

\subsection{Objetivos específicos} \label{subsec:objetivos_especificos}

Los objetivos específicos para este proyecto son:\\

\begin{itemize}
\item
Implementar los principales algoritmos de extracci\'on de caracter\'isticas ac\'usticas  (MFCC, LPC, PLP).
\item
Comprender el funcionamiento de los principales algoritmos de clasificaci\'on usados en el reconocimiento de locutor (GMM, VQ).
\item
Explorar el espacio param\'etrico de las t\'ecnicas de extracci\'on de caracter\'isticas ac\'usticas.
\end{itemize}
\medskip


\section{Planteamiento del problema} \label{sect:problema}

En algunos casos teniendo claros los criterios a considerar para el diseño del sistema, puede no ser tan f\'acil decidir la configuraci\'on de los m\'odulos que hemos mencionado, lo que puede resultar en un tiempo de desarrollo prolongado en donde se hagan pruebas con distintas configuraciones buscando la mas adecuada, o en un sistema no \'optimo usando una configuraci\'on que no est\'e suficientemente justificada.\\

La intención de este trabajo, es presentar un resumen de varias caracterí­sticas acústicas en un espacio variado de parámetros contrastadas con algunos de los modelos de clasificación mas usados, con la finalidad de poder tomar una decisión a priori justificando algunos criterios de desempeño. 
\medskip


\section{Motivación} \label{sect:motivacion}

Uno de los problemas que surge en el diseño de un sistema de reconocimiento de locutor, es en la fase de la extracci\'on de caracter\'isticas, ya que durante el desarrollo de esta disciplina se han utilizado varias t\'ecnicas que hacen enf\'asis en distintas propiedades de la voz humana y de la percepci\'on de la voz, as\'i como en m\'etodos computacionales de procesamiento de señales digitales, esto por la necesidad de reducir la dimensi\'on de los datos, desechando la mayor cantidad de informaci\'on posible y a su vez conservado informaci\'on suficiente y relevante que nos ayude a hacer una correcta diferenciaci\'on entre los usuarios.\\

La implementaci\'on de estas t\'ecnicas de extracci\'on de caracter\'isticas se componen de varias fases, en donde se aplican funciones que van transformado la señal de manera que sea mas f\'acil extraer informaci\'on relevante. Cada una de estas funciones requiere el ajuste de un conjunto de par\'ametros, lo que genera la necesidad de realizar un ajuste de un conjunto de m\'ultlipes par\'ametros, esto con la finalidad de optimizar el funcionamiento en varios \'ambitos, como lo son: tener una fiel representaci\'on de la señal, una reducci\'on considerable de la dimensi\'on de los datos y en algunos casos se requiere que el tiempo de procesamiento no sea demasiado largo, esto es importante ya que un mal ajuste de los par\'ametros puede resultar en una mala respuesta del sistema.\\

Como se ha mencionado anteriormente, despu\'es de la extracci\'on de caracter\'isticas, otro m\'odulo fundamental en cualquier sistema de reconocimiento de locutor es el de reconocimiento de patrones, ya que es el que compara las caracter\'isticas extraidas y permite diferenciar la identidad de los distintos usuarios. En esta etapa tambi\'en se tiene que tomar la decisi\'on de cual es el algoritmo de clasificaci\'on mas eficiente, tambi\'en se pueden emplear distintos criterios para decidir cual es mejor, como lo es: que  sea el mas preciso, que sea f\'acil de implementar o de interpretar o que sea r\'apido.\\

Por lo anterior, se considera que una buena elecci\'on del tipo de caracter\'isticas que se van a extraer de la señal de voz, un ajuste \'optimo de los par\'ametros, en conjunto con una buena elecci\'on del algoritmo de clasificaci\'on, ayuda a generar un sistema mas robusto.\\




\section{Hipótesis} \label{sect:hipotesis}

Se asume que en distintas aplicaciones se tienen objetivos especí­ficos mas allá de hacer una correcta clasificación, como podrí­an ser la velocidad en el procesamiento de la señal de voz, la sencillez en la implementación o el espacio de almacenamiento. La hipótesis es que podemos encontrar las caracterí­sticas acústicas con el ajuste de parámetros mas adecuado según los objetivos y necesidades particulares para cada aplicación.

\section{Estructura de la tesis} \label{sect:estructura}

La forma en que se exponen los temas en este trabajo es la siguiente: En el ca\'itulo 2 se presenta de forma breve y general la teor\'ia del reconocimiento de locutor, en el cap\'itulo 3 se exponen temas relacionados con procesamiento de audio y las t\'ecnicas de representaci\'on de la señal ac\'usticas que ser\'an utilizadas posteriormente, en el cap\'itulo 4 se presentan los clasificadores que se utilizaran, as\'i como las m\'etricas con que se evaluar\'an los resultados, en el cap\'itulo 5 se presenta el estado del arte en reconocimiento de locutores, en el cap\'itulo 6 se describe la metodolog\'ia que se utilizar\'a, en el cap\'itulo 7 se presentan los resultados y la discusi\'on y finalmente en el cap\'itulo 8 se dan las conclusiones del trabajo y se proponen posibles trabajos futuros.\\



% Capítulo sobre reconocimiento de audio
% Marco Teorico.
\chapter{Reconocimiento de locutor} \label{chap:Reconocimiento de locutor}

\section{Introducci\'on}

La forma mas natural que tenemos para comunicarnos es hablando, ya que de esta manera es como transmitimos nuestras ideas cotidianamente. Con los avances tecnol\'ogicos y conforme los dispositivos electr\'onicos se integran mas en nuestra vida y en la sociedad, cobra mayor relevancia acceder a estos sistemas a trav\'es del habla.\\

El objetivo de la comunicaci\'on es transmitir informaci\'on, la señal de voz logra transmitir informaci\'on en muchos niveles, en un nivel primario, un porcentaje de la informaci\'on es intencional, es decir el interlocutor activamente articula un mensaje concreto a trav\'es de la concatenaci\'on de fonemas de un lenguaje espec\'ifico. Sin embargo en un nivel secundario, otro porcentaje de la informaci\'on transmitida, es propia del interlocutor, como lo es el sexo, edad, estado de \'animo y de salud, incluso referencias geogr\'aficas, esta informaci\'on tambi\'en se transmite a trav\'es del habla de manera natural, y en la mayor\'ia de los casos el interlocutor no realiza ningun esfuerzo adicional o consciente para transmitirla.\\

El proceso por el cu\'al se transmite la informaci\'on, comienza con la representaci\'on del mensaje de forma abstracta en la mente del interlocutor, esta se convierte en señales neuronales que controlan los mecanismos articulatorios (lengua, labios, cuerdas vocales, etc.), estos se mueven en respuesta a dichas señales, produciendo una secuencia de gestos, que finalmente producen una señal ac\'ustica en la cual se encuentra el mensaje codificado, m\'as otra informaci\'on que como ya habiamos mencionado, es propia del locutor.\\

Una forma de caracterizar la comunicaci\'on oral, es en el contexto de la teor\'ia de la informaci\'on, esta nos dice que el habla puede representarse en t\'erminos del contenido del mensaje. Por ejemplo, se estima que los humanos podemos producir en promedio 10 fonemas por segundo \cite{rabiner1987}, tomando en cuenta que cada lenguaje tiene un conjunto de entre 30 y 50 fonemas, podemos representar cada fonema con un n\'umero binario de seis bits, esto nos da un estimado de la velocidad en la que se transmite la informaci\'on de 60 bits por segundo e promedio.\\

La representaci\'on de la señal del habla debe ser tal que sea f\'acil para un humano o una m\'aquina poder extraer la informaci\'on, as\'i que en realidad una buena representaci\'on de la señal (que no tome en cuenta solamente la transmisi\'on del mensaje) en el contexto del reconocimiento de locutor, puede llegar a requerir entre 500 y un mill\'on de bits por segundo \cite{rabiner1987}. Aunque la caracterizaci\'on que nos brinda la teor\'ia de la informaci\'on es clara y cuantificable, en la pr\'actica es mas \'util apoyarse en una caracterizaci\'on de la onda ac\'ustica, ya sea en el dominio del tiempo, o en el de las frecuencias.\\

Como se ha mencionado anteriormente, la señal de voz en su representaci\'on ac\'ustica ha resultado muy \'util en aplicaciones pr\'acticas, ya que transmite informaci\'on multidimensional, esta propiedad hace posible que se pueda extraer informaci\'on variada, en particular hay tres tipos principales de informaci\'on que es posible extraer: el mensaje hablado, el lenguaje y la identidad del locutor. De esta distinci\'on se generan tres tipos de sistemas de reconocimiento: el reconocimiento de voz, reconocimiento de lenguaje y el reconocimiento de locutor. De estos tres el mas popular es el reconocimiento de voz, sin embargo el reconocimiento de locutor va cobrando mayor relevancia y cada vez recibe mas atenci\'on. Esta tesis se centra exclusivamente en el reconocimiento de locutor.\\



\section{Conceptos fundamentales}

El reconocimiento de locutor es el proceso mediante el cual se extrae, caracteriza y clasifica la informaci\'on de la señal de voz con la finalidad de reconocer la identidad de la persona que produce dicha señal. En algunos contextos y dependiendo de la finalidad del sistema se le denota como biometr\'ia de voz, esto es por que se usan propiedades fisiol\'ogicas del individuo (en este caso el tracto vocal) para reconocer su identidad, y aunque en algunos casos es comprensible que se defina de esta manera, en la mayor\'ia de casos es preferible el t\'ermino de reconocimiento de locutor, por ser mas general. Cabe destacar el valor del reconocimiento de locutor como biometr\'ia, por sus posibilidades de realizarse a distancia a trav\'es de una linea telef\'onica, esto le da mayor flexibilidad y relevancia.\\

En la literatura se manejan varias divisiones para el reconocimiento de locutor, en \cite{beigi2011} el autor maneja seis posibles ramas que a su vez divide en dos grupos: simples y compuestas. En el primer grupo incluye: la identificaci\'on, la verificaci\'on y la clasificaci\'on, mientras que en el segundo grupo incluye: la segmentaci\'on, la detecci\'on y el seguimiento. A las ramas del primer grupo las considera simples por ser autocontenidas y realizar una funci\'on espec\'ifica, mientras que al segundo grupo las considera compuestas por apoyarse en una o m\'as manifestaciones de las ramas simples con ayuda de t\'ecnicas extra.\\ 

Otros autores \cite{reynolds2002} manejan tres ramas: identificaci\'on, clasificaci\'on y verificaci\'on las cuales corresponden al grupo que se denota como simple en \cite{beigi2011}. En la mayor parte de la literatura solo se manejan dos distintas ramas: la identificaci\'on de locutor y la verificaci\'on de locutor, en esta tesis tomaremos este \'ultimo enfoque, por considerarlo el mas pr\'actico, por tener el grado de especificidad suficiente para el problema que se aborda y por que las ramas que quedan fuera pueden ser consideradas como subramas de la identificaci\'on de locutor. \\

Actualmente la verificaci\'on de locutor, es la rama mas popular del reconocimiento de locutor, debido a su importancia en aplicaciones de seguridad y en control de accesos, cabe mencionar que la implementaci\'on de estos sistemas presenta menos dificultades que las que puede presentar un sistema de identificaci\'on de locutor.

\subsection{Verificaci\'on de locutor}
La verificaci\'on de locutor (o tambien denominada autenticaci\'on de locutor o biometr\'ia de voz) consiste en que un individuo, llamado locutor de prueba, se identifica mediante alg\'un mecanismo alternativo a la voz, esto puede ser un id de registro, un nombre de usuario, una contraseña, etc. Este ID esta asociado a un modelo ac\'ustico de locutor en la base de datos, a este modelo se le denomina modelo de locutor objetivo, el locutor de prueba ingresa su señal de voz y esta es comparada con el modelo objetivo, con la finalidad de verificar si la señal del locutor de prueba puede pertenecer al locutor objetivo, y de ser asi se realiza la verificaci\'on de manera positiva, de lo contrario no se confirma la verificaci\'on del locutor de prueba.\\

Debido a la compleja naturaleza de la señal de voz, resulta impr\'actico hacer una comparaci\'on exclusivamente entre el locutor de prueba y el locutor objetivo, ya que se puede tener una medici\'on del parecido que guardan ambas señales, pero no se tiene una referencia del nivel de parecido que ser\'a aceptable para el sistema. Para resolver esto se suele introducir un tercer modelo que ayuda a contrastar la similitud entre el modelo de prueba y el modelo objetivo, dando una referencia del nivel que similitud que pueda ser significativo en la verificaci\'on. Este modelo de referencia debe tener ciertas propiedades que logren generalizar las caracter\'isticas si no de toda la poblaci\'on, al menos de un grupo representativo de esta. Existen varios enfoques para contrastar el nivel de similitud introduciendo un tercer modelo, a continuaci\'on se mencionan los dos enfoques principales que se manejan en la bibliogra\'ia.\\

Uno de los m\'etodos mas populares es el  UBM (Universal Background Model por sus siglas en ingl\'es) este modelo se contruye con datos de una gran parte de la poblaci\'on, de esta manera se mide la similitud que el locutor de prueba guarda con el locutor objetivo y se contrasta con la similitud que guarda con el resto de la poblaci\'on, de tal manera que si la primera es mayor que la segunda, se confirma la verificaci\'on, de lo contrario se rechaza.\\

Otro m\'etodo muy popular es el modelo de cohortes, en este m\'etodo, se contruye el modelo de contraste con  individuos de la poblaci\'on que guardan similitud con el locutor objetivo, se realiza un contraste similar al del caso anterior para confirmar la verificaci\'on. Este metodo a pesar de que utiliza una poblaci\'on menor para el modelo de contraste, se puede decir que es mas robusto, ya que el modelo de contraste sera muy parecido al modelo del locutor objetivo.\\

En ambos casos, como se puede apreciar, para realizar la verificaci\'on de locutor, solo se necesita realizar dos comparaciones, esto los convierte en un sistema de f\'acil escalamiento, ya que el computo necesario para su realizaci\'on se mantiene constante independientemente del n\'umero de usuarios registrados.\\

\subsection{Identificaci\'on de locutor}
La identificaci\'on de locutor es el proceso por el cu\'al se encuentra la identidad de un locutor desconocido. A diferencia de la verificaci\'on de locutor en que se contrasta el registro de entrada solamente con dos modelos (el modelo del usuario que se afirma ser y el modelo de contraste) el proceso de identificaci\'on de locutor conlleva una comparaci\'on de uno a muchos, esto por que el modelo de entrada se contrasta con cada uno de los modelos registrados en el sistema, se calcula el nivel de similitud y se reporta el modelo que obtenga la similitud mas alta.\\

Existen dos tipos de sistemas de indentificaci\'on de locutor, los sistemas de conjunto cerrado y los sistemas de conjunto abierto. Los sistemas de conjunto cerrado son los mas sencillos, en estos sistemas el audio del locutor de prueba es contrastado con cada uno de los locutores registrados, y el sistema asigna la identidad del locutor con el que se registra la mayor similitud, cabe mencionar que es posible que el locutor de prueba no se encuentre en el conjunto de los usuarios registrados, a\'un as\'i el sistema devuelve un ID de alguno de los usuarios registrados, en algunos contextos esto puede que no sea lo ideal, los sistemas de conjunto abierto resuelven este problema.\\ 

Los sistemas de conjunto abierto pueden verse como una combinaci\'on de un sistema de identificaci\'on de locutor de conjunto cerrado y un sistema de verificaci\'on de locutor, ya que primero se hace una comparaci\'on uno a muchos, se elige el perfil con el que se obtenga la mayor similitud y posteriormente se realiza una verificaci\'on. Si la verificaci\'on se realiza de manera positiva se se acepta la identidad encontrada por la identificaci\'on de conjunto cerrado, en cambio si la verificaci\'on es rechazada, el sistema no acepta ning\'un ID como valido y se considera que el locutor de prueba no se encuentra registrado en el sistema.\\

Un sistema de verificaci\'on es an\'alogo a un sistema de identificaci\'on de locutor de conjunto abierto en el caso l\'imite en que s\'olo hay un usuario registrado, en el sentido que ambos tienen la misma complejidad. En general un sistema de identificaci\'on de locutor es mas complejo que uno de reconocimiento de locutor, ya que te\'oricamente se compara el locutor de prueba contra todos los modelos de locutores de la base de datos, esto hace que la complejidad crezca de forma lineal conforme el n\'umero de usuarios registrados aumenta.\\

\subsection{Fases del reconocimiento de locutor}

La mayor\'ia de sistemas de aprendizaje supervisado, constan de al menos dos etapas que son fundamentales para su funcionamiento, la etapa de entrenamiento que es necesaria antes de poder hacer un uso efectivo del modelo y la etapa de generalizaci\'on, en donde se usa el modelo para realizar la tarea para la cu\'al fu\'e entrenado. De igual manera los sistemas de reconocimiento de locutor se componen de dos fases, inicialmente es necesaria una fase de registro o enrolamiento de locutores, es an\'aloga a la etapa de entrenamiento, para que el sistema pueda realizar el reconocimiento, necesita saber que es lo que tiene que reconocer, posteriormente viene la fase de reconocimiento o verificaci\'on de locutores, es la tarea para la cu\'al fu\'e diseñado el sistema.\\

En la fase de registro dependiendo del tipo de reconocimiento de locutor y la modalidad, los usuarios ingresan alguna frase o se usa alguna grabaci\'on de su voz y se asigna un c\'odigo \'unico de identificaci\'on de usuario, tambi\'en llamado ID, de esta grabaci\'on se extraen caracter\'isticas ac\'usticas que seran utilizadas para entrenar el algoritmo y generar un modelo de voz del usuario, finalmente, este modelo es almacenado en la base de datos mapeado a la identidad del usuario.\\

En la fase de reconocimiento o verificaci\'on, se toma la señal de voz de un usuario de identidad desconocida, y se aplica la misma t\'ecnica de extracci\'on de caractr\'isticas ac\'usticas que ha sido empleada en la fase de registro, posteriormente estas caracter\'isticas se comparan con los usuarios registrados en la base de datos y se mide la similitud que guarda con cada uno de los modelos registrados y dependiendo de esta se realiza el reconocimiento.\\
 
\subsection{Modalidades del reconocimiento de locutor}
El reconocimiento de locutor puede ser implementado de distintas maneras, cada una de estas formas puede aportar alg\'un beneficio a cambio de imponer requerimientos extra, estas modalidades dependen del contexto, de la flexibilidad en la captaci\'on de la señal de voz, del tipo de reconocimiento que se vaya a implementar y de las necesidades del sistema. La verificaci\'on de locutor es la rama que mas se beneficia, por ser la que se da en un entorno mas controlado, y debido a que algunos de los requerimientos se pueden implementar de forma natural en un sistema de verificaci\'on. Con la finalidad de mejorar la calidad en el reconocimiento o reforzar la seguridad del sistema, estas modalidades exploran formas de incorporar otras fuentes de informaci\'on en conjunto a la señal ac\'ustica.\\ 

\textbf{\large Reconocimiento de locutor independiente de texto}\\

El reconocimiento de locutor independiente de texto es la modalidad mas versatil, esta modalidad que puede ser usada en todas las ramas del reconocimiento de locutor. Un sistema de reconocimiento de locutor independiente de texto, toma en cuenta \'unicamente las caracter\'isticas vocales de cada usuario, y dependiendo de su configuraci\'on no se apoya en ning\'un tipo de informaci\'on extra, en algunos casos incluso el lenguaje es de poca relevancia para el sistema.\\ 

En estos sistemas el texto que se usa en la etapa de registro difiere del texto con el que se compara durante la etapa de verificaci\'on, es por esto que los sistemas de identificaci\'on de locutor son independientes del texto, ya que requieren poca coperaci\'on por parte del usuario, incluso se puede hacer un enrolamiento sin el conocimiento del usuario, esto es de gran utilidad en aplicaciones forenses.\\

Por otro lado, en estos sistemas se requiere un tiempo de registro mas amplio, ya que es necesario cubrir el mayor n\'umero posible de fonemas con la finalidad de aumentar la probabilidad de tener la referencia en la etapa de reconocimiento. Otro problema que se puede encontrar en estos sistemas es para sistemas de verificaci\'on el hecho de poder ingresar cualquier sucesi\'on de palabras puede presentar deficiencias en seguridad.\\  

\textbf{\large Reconocimiento de locutor dependiente de texto}\\

En un sistema dependiente de texto, el texto en la fase de verificaci\'on es el mismo que el utilizado en la fase de registro, o el sistema espera recibir un texto predertiminado en la fase de verificaci\'on. Existen varias formas en que se determina el texto que se solicita. El texto puede ser com\'un para todos los usuarios, es decir, que a los usuarios se les solicite ingresar una frase com\'un. Otra forma es que el usuario elija o se le asigne una contraseña, lo cu\'al puede reforzar la seguridad del sistema. Tambi\'en hay sistemas basados en conocimiento, que explotan informaci\'on del usuario como podr\'ia ser el nombre, su direcci\'on o n\'umero telef\'onico. Finalmente existen sistemas que solicitan al usuario que pronuncia alguna frase de verificaci\'on que puede haber sido o no registrada en la fase de enrrolamiento, estos sistemas pueden estar reforzados por un sistema de reconocimiento de voz.\\

Los sistemas de reconocimiento de locutor dependientes de texto son implementados exclusivamente para sistemas de verificaci\'on de locutor, por la naturaleza de estos sitemas es f\'acil solicitar a los usuarios que registren alguna frase predeterminada, ya que la fase de registro se da en un ambiente controlado y el usuario se registra voluntariamente. En cambio, para sistemas de identificaci\'on de locutor resulta impr\'actico e incluso imposible solicitar texto a los usuarios, ya que se dan en un entorno de habla mas natural y en la mayoria de los casos los usuarios no tienen conocimiento de que estan registrados en el sistema.\\ 

\section{Sistemas de reconocimiento de locutor}

\subsection{Arquitectura}
Un sistema de reconocimiento de locutor (figura 2.1) consta de varios m\'odulos que se agrupan en tres componentes b\'asicos: procesamiento de la señal para la extracci\'on de caracter\'isticas, el modelo de locutores y el reconocimiento de patrones. El procesamiento de la señal consiste en captar la señal de voz, digitalizarla, y procesar la señal para reducir el ruido y remover silencios, posteriormente se realiza un procesamiento mas complejo el cual tiene como finalidad extraer caracter\'isticas ac\'usticas propias de cada locutor. El modelo de locutores se construye con la informaci\'on que se extrae de la señal, puede ser un modelo estad\'istico o un modelo construido por un algoritmo de aprendizaje de m\'aquina, en el reconocimiento de patrones se realiza una comparaci\'on de patrones entre los modelos construidos anteriormente y una nueva señal,  que permita al sistema tomar una decisi\'on sobre la identidad del locutor.\\ 

\begin{figure}[H]
	\begin{center}
	\includegraphics[scale=0.45,type=png,ext=.png,read=.png]{imagenes/diagrama1} \\
	\caption{Diagrama de un sistema gen\'erico de reconocimiento de locutor.}
	\label{fig:diag_recon_locutor}
	\end{center}
\end{figure}

Estos m\'odulos pueden tener distintas configuraciones dependiendo del diseño del sistema, como se ha mencionado anteriormente el reconocimiento de locutor se divide en dos ramas, identificaci\'on de locutor y verificaci\'on de locutor. La diferencia mas notoria entre ambos sistemas se da en la etapa de comparaci\'on de patrones, ya que para el caso de la verificaci\'on se realizan solamente dos comparaciones, mientras que para el caso de la identificaci\'on se compara con todos los modelos de locutor registrados en la base de datos.\\

En la figura 2.2 se muestra la arquitectura de un sistema de verificaci\'on de locutor, en estos sistemas el usuario ingresa la señal de voz y un ID, el sistema accede a la base de datos para obtener el modelo de locutor registrado con el ID, procesa la señal de voz, extrae las caracter\'isticas ac\'usticas y las compara con el modelo del ID y con un modelo universal o de cohortes, con esta informaci\'on decide si la señal de voz pertenece al ID ingresado y acepta o rechaza la verificaci\'on.\\

\begin{figure}[H]
	\begin{center}
	\includegraphics[scale=0.45,type=png,ext=.png,read=.png]{imagenes/diagrama2} \\
	\caption{Diagrama de un sistema de verificaci\'on de locutor.}
	\label{fig:diag_verif_locutor}
	\end{center}
\end{figure}

En la figura 2.3 se muestra la arquitectura de un sistema de identificaci\'on de locutor, aqu\'i el usuario ingresa la señal de voz, se extraen las caracter\'isticas y se comparan con todos los modelos de la base de datos, y dependiendo si el sistema es de conjunto cerrado o abierto, elige el ID  del usuario con mayor similitud o concluye que la señal de voz no pertenece a ning\'un usuario registrado.\\

\begin{figure}[H]
	\begin{center}
	\includegraphics[scale=0.45,type=png,ext=.png,read=.png]{imagenes/diagrama3} \\
	\caption{Diagrama de un sistema de identificaci\'on de locutor.}
	\label{fig:diag_clasif_locutor}
	\end{center}
\end{figure}

A continuaci\'on se expone la fase de registro y las caracter\'isticas de los tres m\'odulos principales.\\


\subsection{Registro}
La primer fase en un sistema de reconocimiento de locutor es la de registro, esta fase se compone de tres etapas, en la primera se obtienen grabaciones de voz del usuario que ser\'a registrado en el sistema, posteriormente se extraen caracter\'isticas de la señal de voz para tener una representac\i\'on mas adecuada, y finalmente se construye un modelo para cada usuario que ser\'a almacenado en la base de datos. El objetivo de esta fase es el de obtener suficiente informaci\'on de cada usuario que le permita al sistema construir un modelo para realizar la tarea de reconocimiento con un bajo de nivel de error y a su vez, no sea necesaria demasiada informaci\'on resultando en una cantidad impr\'actica para su captura.\\

Esta fase funciona de forma distinta dependiendo del sistema, teniendo distintas caracter\'isticas para los sistemas de verificaci\'on y los de identificaci\'on, de la misma manera pose ventajas y retos propios para cada modalidad ya sea independientes o dependientes de texto.\\

En un sistema independiente de texto, existe la ventaja de que no se impone ninguna restricci\'on al usuario e incluso se puede captar la señal de su voz sin que el usuario este consciente de esto, esto a su vez representa un reto, ya que puede darse el caso extremo de que el registro no cuente con una variedad de fonemas suficientes para lograr el reconocimiento, lo que puede resultar en niveles altos de error o para compensar esto puede ser necesario que los tiempos de registro sean mayores. A pesar de esto, cabe recordar, como ya se hab\'ia mencionado anteriormente, los sistemas independientes de texto son la \'unica modalidad que resulta viable para todos los tipos de reconocimiento de locutor.\\

En un sistema dependiente de texto, la misma frase que se usa en la fase de registro, ser\'a usada para realizar la verificaci\'on, esto lo convierte en una modalidad de las mas simples, ya que no hay tomar en cuenta muchas consideraciones, la frase con la que se registra el usuario puede ser elegida por el mismo, o alguna frase que le solicite el sistema, la frase que sea utilizada es indiferente para la tarea de verificaci\'on ya que ser\'a usada la misma frase posteriormente, y la elecci\'on de esta depender\'a de otros factores.\\

El sistema de texto solicitado, es un sistema dependiente de texto que resulta mas complejo que el anterior, en esta modalidad el sistema le solicita al usuario que ingrese una frase espec\'ifica en la fase de registro que no es necesariamente la misma frase que solicitar\'a el sistema en la fase de verificaci\'on. Una forma de implementar este tipo de sistemas es anticipar todos los posibles fonemas que se usaran en la fase de verificaci\'on y cubrirlos en la fase de registro. Una forma mas com\'un, seria solicitando frases personalizadas para cada usuario, en donde se solicite informaci\'on espec\'ifica de cada usuario, como informaci\'on personal, o asignarle un conjunto de palabras claves a cada usuario.\\

\subsection{Procesamiento de la señal de voz}

El procesamiento de la señal de voz es uno de los pasos mas importantes en cualquier sistema de reconocimiento de locutor,  esta etapa tiene como finalidad extraer la informaci\'on deseada de la señal de voz. El proceso comienza captando la señal ac\'ustica generalmente con un micr\'ofono, convirtiendola en una señal el\'ectrica, este proceso requiere la aplicaci\'on de un filtro para evitar el efecto de alising, esto se logra cortando todas las frecuencias que se encuentren por encima de un medio de la frecuencia de sampleo tambi\'en llamada frecuencia de Nyquist.\\

Una vez que se tiene filtrada la señal anal\'ogica, se digitaliza usando un convertidor A/D (anal\'ogico/digital), los convertidores A/D actuales pueden captar la señal a frecuencias de sampleo de 96,000 muestras por segundo, con una resoluci\'on de hasta 24 bits, aunque com\'unmente se suele utilizar una frecuencia de sampleo de 44,100 muestras por segundo con resoluci\'on de 16 bits. Esto quiere decir que por cada muestra tendremos 16 bits, y por cada segundo tendremos 44,100 muestras, lo que da un total de 705,600 bits por segundo, esta es una cantidad de informaci\'on muy grande considerando que podemos producir 10 fonemas por segundo.\\

Esta señal digitalizada en la forma en que se encuentra resulta poco \'util para la tarea de reconocimiento de locutor, en primer lugar la cantidad de informaci\'on resulta impr\'actica ya que hay mucha informaci\'on inecesaria y redundante, un ejemplo son los fragmentos de silencio y la frecuencia de sampleo, la frecuencia de sampleo de 44,100 Hz capta frecuencias hasta 22,050 Hz, aqu\'i hay informaci\'on  que resulta inutil ya que la voz humana se extiende de los 50 Hz a los 8,000 Hz, por lo tanto la señal puede reducirse eliminando los fragmentos de silencio y cambiando la frecuencia de sampleo a 16,000 muestras por segundo, reduciendo un poco la cantidad de informaci\'on.\\

Lo anterior a pesar de ser necesario para tener una señal mas limpia y manejable, no es suficiente, la señal sigue teniendo un flujo de informaci\'on muy grande, adem\'as, la forma en que esta representada la señal, no es la mas \'optima, la señal se encuentra representada como variaci\'on de presi\'on de aire en el tiempo, esto no es muy \'util para el reconocimiento de locutor, ya que lo que diferencia los distintos tonos de voz de las personas, no es la forma en que modifican la presi\'on de aire, lo que realmente diferencia los tonos de voz, es el contenido espectral. Es por esto que se necesita una representaci\'on de la señal en donde se enfatize el espectro arm\'onico.\\

\begin{figure}[H]
	\begin{center}
	\includegraphics[scale=0.5,type=png,ext=.png,read=.png]{imagenes/diagrama4} \\
	\caption{Captura y procesamiento de la señal.}
	\label{fig:diag_verif_locutor}
	\end{center}
\end{figure}

Las t\'ecnicas m\'as comunes que se utilizan en el reconocimiento de locutor hacen \'enfasis en las cualidades arm\'onicas de la señal, en el caso de el banco de filtros de frecuencias Mel y los coeficientes cepstrales de frecuencias Mel, se aplica la transformada r\'apida de Fourier, que es la version computacional de la transformada de Fourier, lo que hace este algoritmo es que transforma la señal del \'ambito del tiempo al \'ambito de las frecuencias, posteriormente aplica un conjunto de filtros que buscan emular la forma en que el oido humano capta el sonido, estos son los procesos m\'as importantes del c\'alculo, en el cap\'itulo siguiente se presenta todo el procedimiento.\\

Otras dos t\'ecnicas tambi\'en comunes son el c\'odigo de predicci\'on lineal y la predicci\'on de percepci\'on lineal, estas t\'ecnicas tambi\'en hacen \'enfasis en el material arm\'onico, aunque sin el uso de la transformada de Fourier, en estas representaciones se modela el tracto vocal encontrando las resonancias o formantes de la señal, lo que resulta en una representaci\'on de la envolvente del espectro arm\'onico, en el caso de la predicci\'on de percepci\'on lineal a la envolvente se le aplica un banco de filtros que emulan la escucha humana, estas dos t\'ecnicas tambien se desarrollan mas a detalle en el siguiente cap\'itulo.\\

Estas t\'ecnicas no solo enfatizan el espectro arm\'onico de la señal, si no que reducen considerablemente su dimensi\'on, lo que contribuye a tener una señal con informaci\'on mas compacta, enfocada al problema que se busca resolver y manejable tanto en tiempo de procesamiento como en espacio de almacenamiento.

\subsection{Modelo de locutor}

En la fase de registro despu\'es de realizar la extracci\'on de caracter\'isticas de la señal, se construye un modelo de locutor propio del usuario utilizando el vector de caracter\'isticas, el objetivo de crear un modelo de locutores es el de poder asociar a cada usuario con un identificador que permita al sistema diferenciarlo de los dem\'as usuarios. Se considera que algunos de los atributos deseables que debe poseer son: Bases te\'oricas s\'olidas que puedan dar claridad sobre el comportamiento del modelo, que sea generalizable sobre nuevos datos, es decir que no sobre ajuste los datos de registro y que sea pr\'actico en t\'erminos de tiempo de procesamiento y no requiera demasiado espacio de almacenamiento. Este proceso por estar en el centro de cualquier aplicaci\'on de reconocimiento de locutor puede ser considerado como el m\'as importante, es por esto que debe ser abordado con las consideraciones suficientes en el diseño del sistema.\\

Existen varias t\'ecnicas que han sido empleadas en el reconocimiento de locutor y la elecci\'on de una de estas depende de las caracter\'isticas de la señal, del rendimiento esperado del sistema, la facilidad para entrenarlo y actualizarlo y principalmente de la modalidad del sistema de reconocimiento. A continuaci\'on se enlistan algunas de las t\'ecnicas mas com\'unmente usadas.\\

\textbf{Tiempo din\'amico de deformaci\'on.- } En este modelo se crea una plantilla, que consiste en un conjunto de vectores pertenecientes a alguna frase fija que se obtiene en la etapa de registro, posteriormente en la etapa de reconocimiento la misma frase es ingresada por el usuario, este algoritmo alinea ambas frases para que su longitud sea la misma, este paso es necesario debido a la variabilidad del habla, posteriormente se mide la similitud que guardan entre si, esta t\'ecnica es casi exclusiva para los sistemas de verificaci\'on dependientes de texto, ya que en estos casos su rendimiento es mayor.\\

\textbf{Modelos ocultos de Markov.- } En estos modelos se codifica la evoluci\'on temporal de las caracter\'isticas y se realiza un modelo estad\'istico de su variaci\'on, de esta manera se obtiene una representaci\'on estad\'istica de la manera en que cada usuario produce sonido. Durante la fase de registro se obtienen los par\'ametros del modelo y en la fase de reconocimiento se estima la verosimilitud de que la secuencia ingresada pertenezca al modelo. En sistemas independientes de texto se suele usar un modelo de un solo estado, conocido como modelo de mezcla de Gaussianas (Gaussian Mixture Model), en este modelo se asigna una distribuci\'on normal a cada usuario y se calculan sus par\'ametros en la fase de registro, posteriormente en la fase de reconocimiento para el caso de identificaci\'on de locutor se busca la distribuci\'on en la que la señal guarda la m\'axima verosimilitud.\\

\textbf{Cuantificaci\'on vectorial.- } Esta es una t\'ecnica de clusterizaci\'on en la que se mapean los vectores de un espacio vectorial a un conjunto finito de regiones en dicho espacio, es un algoritmo de compresi\'on de datos con perdida o tambi\'en llamada compresi\'on irreversible. En el contexto del reconocimiento de locutor el algoritmo utiliza los vectores de caracter\'isticas de la fase de registro para generar una partici\'on del mismo tamaño que el n\'umero de usuarios registrados, cada clase es representada por el centroide de la regi\'on a la que pertenece llamado vector c\'odigo, el conjunto de todos los vectores c\'odigo es llamado libro de c\'odigos, el cu\'al sirve como modelo de los locutores, este conjunto es significativamente menor al conjunto de vectores de la fase de registro. En la etapa de reconocimiento se utiliza una m\'etrica que mide la similitud entre los vectores de entrada y cada vector c\'odigo, t\'ipicamente es la medida de distorsi\'on de cuantificaci\'on entre dos conjuntos de vectores, finalmente se asigna a la clase en que la distorsi\'on es m\'inima.\\

\textbf{Redes neuronales artificiales.- } Las redes neuronales artificiales son una familia de modelos que pueden encontrar relaciones complejas en los datos al mismo tiempo que extraen caracter\'isticas, por esta raz\'on puede que para un sistema de reconocimiento de locutor que utilice redes neuronales resulte inecesaria la etapa de extracci\'on de caracter\'isticas. Gracias a las distintas formas de representaci\'on de la señal de audio, entre los distintos modelos de redes neuronales existen varios que se adaptan a las necesidades del reconocimiento de locutores. La señal de audio por ser una serie de tiempo unidimensional, puede ser modelada por redes neuronales recurrentes o por una red de memoria a corto y largo plazo, ambas arquitecturas est\'an diseñadas para procesar y encontrar relaciones en secuencias temporales. Otra arquitectura es la de las redes neuronales convolucionales que est\'a diseñada para procesar im\'agenes y encontrar relaciones espaciales, si la señal de audio se representa con el espectrograma este puede ser procesado por este tipo de red neuronal. Puede haber otras arquitecturas que funcionen para el procesamiento de señales de voz como las redes neuronales profundas, sin embargo se ha demostrado que en general las redes neuronales artificiales son computacionalmente muy costosas y que en algunos casos los modelos no son generalizables \cite{reynolds2002}.\\

 


\subsection{Reconocimiento de patrones}

El reconocimiento de patrones o comparaci\'on de patrones se encuentra en la fase de reconocimiento, es posterior a la fase de registro ya que es necesario que los modelos de locutores est\'en almacenados. Esta tarea consiste en asignar una puntuaci\'on de coincidencia, la cu\'al es una medida de similitud entre un vector de caracter\'isticas que a\'un no ha sido observado por el sistema y puede o no pertenecer a alguno de los usuarios registrados y alg\'un modelo de la base de datos.\\

Esta medida de similitud depende del modelo de locutores que ha sido implementado en el sistema, en \cite{campbell1997} se menciona que existen dos tipos de modelos: los de plantilla y los estoc\'asticos. Para los modelos de plantilla como el tiempo din\'amico de deformaci\'on y la cuntificaci\'on de vectores, se mide de manera determin\'istica, con m\'etricas de distancia como la distancia Euclidea o la distancia de Mahalanobis, mientras que para los modelos estoc\'asticos como los modelos ocultos de Markov, o la mezcla de Gaussianas se mide de manera probabil\'istica, con medidas de verosimilitud o probabilidad condicional.\\

Esta medida es utilizada para la decisi\'on que tomar\'a el sistema en el reconocimiento, esta desici\'on depende del tipo de reconocimiento del sistema, siendo distinta en el caso de la verificaci\'on al de identificaci\'on y tambi\'en en el diseño del sistema.\\ 

Para el caso de la verificaci\'on de locutor, una vez obtenida la metrica de similitud, el sistema necesita una referencia para saber si esta medida es suficiente para aceptar o rechazar la verificaci\'on, esta referencia esta representada en el diagrama de la figura 2.2 como el umbral, este umbral puede obtenerse de tres formas. El caso mas sencillo es en el que durante el diseño del sistema se define un umbral fijo para todos los usuarios, si la similitud esta por debajo de este umbral se rechaza la verificaci\'on y si es mayor se acepta. Otra forma de obtener el umbral, es con la similitud que guarda la señal con un modelo de referencia, en la literatura se manejan dos formas de construir este modelo, el modelo universal de fondo y el modelo de cohortes, estos modelos lo que buscan es representar las caracter\'isticas generales de una poblaci\'on o de un subconjunto espec\'ifico de la poblaci\'on para obtener un contraste significativo de similitud, si la similitud de la señal con el modelo de veirficaci\'on es mayor que la similitud con el modelo de referencia, la verificaci\'on es aceptada, en caso contrario se rechaza. Dependiendo de las caracter\'isticas del sistema y de la poblaci\'on se recomienda uno sobre el otro.\\



En el caso de la identificaci\'on de locutor, se mide la similitud que la señal guarda con cada uno de los modelos registrados y dependiendo si es un sistema de conjunto cerrado, puede elegir alg\'un usuario con el que guarde la similitud mas alta, o si es un sistema de conjunto abierto, elegir al usuario con mayor similitud o descartarlos a todos y concluir que es un usuario desconocido. Estos sistemas tienen la caracter\'istica que a mayor n\'umero de usuarios registrados el proceso de reconocimiento se prolonga.\\




%La principal forma que tenemos los humanos para comunicarnos es hablando, y la función de comunicarnos es la transmisión de información desde una fuente hacia un receptor, la persona que habla es la fuente y el receptor generalmente es otra persona. Esa información suele presentarserse y representarse en distintas formas. Comienza como impulsos eléctricos en el cerebro, se envian señales a los m\'usculos del tracto vocal para que genere y modele una onda acústica, para el caso de las aplicaciones de reconocimiento de habla el receptor es una máquina que recibe la señal acústica por medio de micr\'ofonos que convierten la onda acústica en impulsos eléctricos, estos pasan por un convertidor anal\'ogico/digital que samplea y quantiza la señal eléctrica conviertiendola en datos que la computadora puede leer, generalmente representado como arreglos de números.\\

%Estos arreglos de números son una representación de la onda acústica, entre la información que contiene esta representación se encuentra el mensaje que quiere enviar el interlocutor, su estado emocional, el leguage que usa, el ambiente acústico que lo rodea, caracterí­sticas propias de los micr\'ofonos y filtros utilizados, entre otras. La información que nos concierne es la información que es propia de la persona que esta hablando. Para poder separar de la señal digitalizada la información que nos interesa, se le da un tratamiento a la señal que entre otras cosas, extrae cualidades de percepción humana, reduce la dimensión de los datos, y deshecha información que no nos interesa.





% Capítulo sobre reconocimiento de audio
% Marco Teorico.
\chapter{Procesamiento y representaci\'on de la señal de voz} \label{chap:na}

\section{Introducci\'on}


El procesamiento de la señal de voz en el reconocimiento de locutor, tiene como finalidad obtener una representaci\'on de la señal de voz que permita diferenciar con la mayor precisi\'on posible entre distintos locutores. Para lograr esto es necesario que la señal de voz pase por una cadena de procesos que pueden englobarse en las siguientes etapas:\\

\textbf{Captura.-} Para poder procesar la voz, es necesario capturarla y digitalizarla, la captura se realiza generalmente con un micr\'ofono mediante un proceso de transducci\'on se convierten las ondas mec\'anicas en voltaje, despu\'es se digitaliza, pasando de voltaje a c\'odigo binario, lo que hace posible el procesamiento computacional.\\

\textbf{Pre-procesamiento.-} Posteriormente a la etapa de captura, el audio tiene que ser preparado para eliminar efectos no deseados o para que sea mas manejable, el objetivo es mejorar la calidad de la señal, en esta etapa se suele eliminar ruidos o distorsiones, eliminar o reducir reverberaciones, separar audio proveniente de distintas fuentes, detecci\'on de la actividad de voz, entre otros.\\ 

\textbf{Extracci\'on de caracter\'isticas.-} Finalmente, se busca extraer par\'ametros de la señal que contengan exclusivamente la informaci\'on necesaria para realizar un an\'alisis que permita resolver un problema o realizar una tarea espec\'ifica, e idealmente se logre descartar el resto de informaci\'on. Estos par\'ametros pueden estar tanto en el dominio del tiempo como en el de las frecuencias y se denomina vector de caracter\'isticas \cite{schuller}.\\
\indent


\subsection{Sonido}
El sonido son ondas mec\'anicas que se propagan a trav\'es de un medio generalmente el aire (aunque puede ser un  l\'iquido o s\'olido) y son producidas por una fuente que vibra. Las vibraciones perturban las mol\'eculas al rededor de la fuente alejandolas y juntandolas en sincronia con las vibraciones, de esta manera se producen pequeñas regiones en el medio en que la presi\'on es menor (rarefacci\'on) y regiones en la que la presi\'on es mayor (compresi\'on). Estas regiones en que se alterna rarefacci\'on y compresi\'on del medio, se propagan desde la fuente en todas direcciones produciendo sonido, la velocidad de propagaci\'on en el aire es de aproximadamente 343.2 m/s, esta es independiente de las caracter\'isticas de la onda y depende exclusivamente del medio y de sus condiciones, viajando mas r\'apido en medios con mayor densidad o mayor temperatura. La interacci\'on entre rarefacci\'on y compresi\'on en el tiempo genera patrones denominados forma de onda, las caracter\'isticas de la forma de onda son fundamentales en el resultado sonoro.\\ 

La forma de onda puede componerse de una unidad de forma mas pequeña que se repite constantemente llamada ciclo, el tiempo que toma completar un ciclo se llama peri\'odo, a este tipo de ondas se les llama ondas peri\'odicas, los sonidos que entran en el rango de la audici\'on humana tienen periodos de entre 0.00005 y 0.05 segundos aproximadamente. La velocidad en que se repite el ciclo es la frecuencia de onda, la unidad de medida de frecuencia son los Hertz (Hz), esta unidad representa el n\'umero de ciclos por segundo, matem\'aticamente la frecuencia es el inverso del periodo, es decir $f = \frac{1}{p}$. El rango de audici\'on humana expresado en Hertz va de los 20Hz a los 20KHz, es decir la frecuencia mas baja en la audici\'on humana es de 20 ciclos por segundo y llega hasta 20 mil ciclos por segundo, siendo esta la mas alta. El rango de audici\'on humana es aproximado ya que depende de varios factores, como las condiciones de escucha, o la edad del escucha tendiendo a reducirse este rango con la edad.\\

Otra caracter\'istica importante de las ondas sonoras es la amplitud, esta es la magnitud del cambio de presi\'on en el medio causado por la rarefacci\'on y la compresi\'on, esta din\'amica genera una cierta fuerza sobre el medio ya que las mol\'eculas se empujan entre si colectivamente. La amplitud se mide en Newtons por metro cuadrado (N/$m^2$), esto es la cantidad de fuerza que es aplicada en una \'area determinada. La m\'inima amplitud audible es de aproximadamente 0.00002 N/$m^2$, mientras que del otro lado del rango una amplitud de 200 N/$m^2$ es el l\'imite cuando un sonido comienza a ser percibido no s\'olo por el oido sino por el cuerpo entero   \cite{dodge}.\\

%Es conveniente diferenciar el concepto de onda sonora del de evento sonoro, siendo el primero una abstracci\'on fenomenol\'ogica del segundo. Las ondas sonoras tienen cuatro propiedades, frecuencia, fase, amplitud y espectro arm\'onico, un evento sonoro tiene dos propiedad mas, la fuente donde se origina, y la duraci\'on, estas dos propiedades ubican al evento sonoro en el espacio y en el tiempo. La fuente debe ser \'unica y por tener una ubicaci\'on espacial, define la posici\'on del evento sonoro, la duraci\'on define la ubicaci\'on temporal del evento sonoro, es decir tiene un punto inicial y final, la frecuencia, fase,  amplitud, espectro arm\'onico e incluso la fuente pueden modificarse en este intervalo, y esta din\'amica es lo que caracteriza al evento sonoro.\\

\subsection{Señales de audio}

El t\'ermino audio es entendido como una representaci\'on del sonido que tiene como finalidad transformar, transmitir, reproducir y almacenar sonidos. Puede ser segmentado en tres categor\'ias: m\'usica, habla y sonido en general, con aplicaciones en m\'usica, telecomunicaciones, procesamiento y s\'intesis de voz y an\'alisis y s\'intesis de sonidos \cite{schuller}.\\

Las señales de audio pueden ser multicanal como es el caso de los sistemas estereo o sistemas 5.1, o en el caso de las señales mono o las señales telef\'onicas, de un solo canal. A excepci\'on de señales de audio muy simples como lo son tonos generados por osciladores, las señales de audio pueden llegar a ser muy complejas como se observa en la figura 3.1 en general no son peri\'odicas ni determin\'isticas, esto quiere decir que la forma de onda y la intensidad cambian constantemente y no es posible representar la señal por medio de una f\'ormula matem\'atica.\\ 

\begin{figure}[H]
	\begin{center}
	\includegraphics[scale=0.45,type=png,ext=.png,read=.png]{imagenes/center_voice} \\
	\caption{Fragmento de señal de voz digitalizada.}
	\label{fig:diag_recon_locutor}
	\end{center}
\end{figure}

Las señales de audio comunmente se encuentran en el dominio del tiempo, es decir como cambios de presi\'on o de voltaje en el tiempo, sin embargo tambi\'en pueden caracterizarse en t\'erminos de su contenido de frecuencias o espectro arm\'onico, esta representaci\'on se obtiene usando la transformada de Fourier \cite{kamen} y visualizarse con un espectrograma de frecuencias, es muy \'util tener ambas representaciones ya que en algunos casos es preferible una sobre la otra, dependiendo del an\'alisis que se desee realizar.\\

La señal puede ser el\'ectrica an\'alogica, esta representaci\'on tiene la caracter\'istica de ser continua, puede ser sintetizada con osciladores o registrada con micr\'ofonos, en un proceso de transducci\'on en el que el sonido pasa de ser fluctuaciones de presi\'on en el aire a fluctuaciones de voltaje el\'ectrico. Tambi\'en puede representarse num\'ericamente a trav\'es de un proceso de muestreo y cuantizaci\'on que se realiza con un convertidor anal\'ogico digital, esta representaci\'on en contraste con la anterior es discreta. Este \'ultimo caso es el que resulta de mayor relevancia en este trabajo, en la siguiente secci\'on se presentan algunas t\'ecnicas b\'asicas de procesamiento digital de audio.\\




\section{Procesamiento de audio digital}

\subsection{Procesamiento de señales digitales}
El procesamiento de señales digitales es un conjunto de t\'ecnicas empleadas para operar matem\'atica y computacionalmente una señal discreta codificada num\'ericamente con la finalidad de modificarla o mejorarla en alg\'un sentido, y que facilite el proceso de extraer, transmitir o almacenar informaci\'on. Esta tecnolog\'ia ha generado un gran imp\'acto en muchas \'areas del conocimiento con m\'ultiples aplicaciones pr\'acticas.\\

El procesamiento de señales digitales surge en los años sesenta, cuando las computadoras comenzaron a estar disponibles, aunque eran muy costosas y con limitaciones, inmediatamente se evidenci\'o su potencial. En esa \'epoca sus aplicaciones eran principalmente para radares militares, exploraci\'on espacial y aplicaciones m\'edicas \cite{smith}, actualmente el procesamiento de señales digitales ha llegado a un nivel de madurez gracias al teorema de muestreo, la transformada r\'apida de Fourier y a la accesibilidad y evoluci\'on de las capacidades de c\'omputo,  sus aplicaciones se encuentran en todos lados, favoreciendo el desarrollo en m\'ultiples disciplinas cient\'ificas, industriales y comerciales, con notable desarrollo en telecomunicaciones, procesamiento de voz y procesamiento de audio en general \cite{li-cox}.\\

Las señales son mediciones o representaciones de fen\'omenos f\'isicos y en su mayor\'ia son de naturaleza continua tanto en magnitud como en el tiempo, para poder aplicar los principios del procesamiento de señales digitales, deben ser convertidas a formato digital, tienen que pasar por un proceso de muestreo y cuantizaci\'on en donde pierden su naturaleza continua. Este proceso toma tiempo y aunque en la mayor\'ia de casos es insignificante, considerando que los sistemas anal\'ogicos realizan sus procesos en tiempo real, esto se convierte en una desventaja del audio digital sobre el an\'alogo, sobre todo en casos donde es necesario trabajar a altas frecuencias o con un amplio ancho de banda, ya que requieren procesar una cantidad mayor de datos. A pesar de esto en general es preferible trabajar con señales digitales, ya que ofrecen mayor precisi\'on, flexibilidad, son menos costosos de implementar y la informaci\'on digital es m\'as f\'acil de almacenar.\\

\subsection{Convertidor de audio anal\'ogico digital}

Para procesar computacionalmente una señal de audio es necesario transformarla de su forma an\'aloga a su forma digital, es decir convertir una señal continua en amplitud y en el tiempo en una señal discreta en amplitud y en el tiempo, esta conversi\'on se realiza con un convertidor de audio anal\'ogico digital.\\
Este proceso se ilustra en la figura() y consta de los siguientes tres pasos:\\

\begin{itemize}
	\item \textbf{Muestreo.- }En este paso se convierte una señal que es continua en el tiempo, en discreta tomando muestras de la señal a intervalos constante de tiempo, es necesario definir la frecuencia de muestreo, es decir el n\'umero de muestras que se capturan cada segundo.
	\item \textbf{Cuantizaci\'on.- }En la cuantizaci\'on se discretiza la amplitud de la señal restringiendo los valores que pueden ser registrados a un conjunto finito y se asigna de este conjunto el valor mas cercano al valor real de la señal, en este paso se tiene que definir la cantidad de valores  que puede tomar la señal digital, esto se define en n\'umero de bits.
	\item \textbf{Codificaci\'on.- }En esta etapa se convierten los valores n\'umericos, que estan en alg\'un formato de enteros o flotantes a un formato binario, generalmente este proceso ocurre simultaneamente al de cuantizaci\'on.
\end{itemize}

Este proceso restringe la cantidad de informaci\'on que ser\'a registrada, y para poder realizar este proceso de forma \'optima hay que tomar en cuenta algunas consideraciones, ya que es necesario entender que informaci\'on se necesita capturar y que informaci\'on puede ser descartada \cite{smith}. Los convertidores de audio anal\'ogico digital comunmente utilizan frecuencias de muestreo que pueden ir desde 44.1 KHz a 192 KHz y profundidad de bits que pueden ir desde 8 hasta 32 bits, estas configuraciones son empleadas en audio de alta fidelidad, en telecomunicaciones suelen emplearse frecuencias de muestreo mas bajas, por lo que son configuraciones que permiten realizar los procesos correctamente.\\

\begin{figure}[H]
	\begin{center}
	\includegraphics[scale=0.45,type=png,ext=.png,read=.png]{imagenes/diagrama05} \\
	\caption{Fragmento de señal de voz digitalizada.}
	\label{fig:diag_recon_locutor}
	\end{center}
\end{figure}

Como se ve en la figura, previo a realizar la conversi\'on de señal an\'aloga a digital,  la señal an\'aloga debe pasar por un filtro pasa bajas y por un m\'odulo de muestreo y retenci\'on para poder discretizar la señal. El filtro tiene como finalidad restringir las frecuencias que ser\'an capturadas ya que frecuencias mayores a un medio de la frecuencia de muestreo no pueden ser interpretadas correctamente. El circuito de muestreo y retenci\'on es un dispositivo an\'alogo que recibe una señal continua y retiene el valor a un nivel constante por un lapso determinado de tiempo, este circuito es necesario en un convertidor anal\'ogico digital, para evitar variaciones en la señal que puedan corromper el proceso de conversi\'on ya que la amplitud de la señal an\'aloga se modifica constantemente y el proceso de conversi\'on toma tiempo, esto puede provocar inconsistencias en el proceso de conversi\'on y generar errores en el registro de la señal.\\

Una vez realizado el procesamiento o almacenamiento de la señal digital, es necesario regresar la señal a su representaci\'on an\'aloga, este proceso se realiza con un convertidor digital anal\'ogico, es el proceso inverso al de digitalizaci\'on, y se realiza interpolando los valores digitales, esta interpolaci\'on puede realizarse a trav\'es de varias t\'ecnicas, la mas com\'un es la conversion de orden cero, aunque tambien puede realizarse una interpolaci\'on lineal o cuadratica \cite{proakis}. En este trabajo s\'olo es relevante la digitalizaci\'on por lo que no se profundizar\'a en la conversi\'on digital anal\'ogica.\\


\subsubsection{Muestreo}
El primer paso del muestreo es la discretizaci\'on de la señal en el tiempo, el voltaje de la señal anal\'oga fluctua constantemente y a cada instante de tiempo su valor cambia, es imposible para la computadora capturar el valor en cada instante, es por esto que se toman registros de la señal en intervalos constantes de tiempo, el resultado del proceso de muestreo es una secuencia de n\'umeros correspondientes al voltaje de la señal anal\'oga en cada tiempo muestreo, este m\'etodo de representar una señal an\'aloga se conoce como modulaci\'on de c\'odigo de pulso (PCM)\cite{dodge}, aunque existen otros m\'etodos para realizar el muestreo, como el muestreo aleatorio, la modulaci\'on de ancho de pulso o el muestreo c\'iclico, el mas com\'un en el procesamiento de voz es el muestreo peri\'odico \cite{beigi2011}. \\

Como se mencion\'o anteriormente el muestreo por modul\'on de c\'odigo de pulso es un muestreo peri\'odico, es decir, el tiempo \begin{math} \Delta \end{math}t que transcurre entre dos registros es fijo y se denomina periodo de muestreo, la frecuencia de muestreo \begin{math} f = \frac{1}{\Delta t} \end{math} es el inverso del periodo de muestreo es expresado en Hertz y representa el n\'umero de registros capturados en un segundo.\\

Una mayor frecuencia de muestreo, resulta en una representaci\'on mas precisa de la señal continua, sin embargo la cantidad de informaci\'on que se tiene que almacenar o procesar ser\'a mayor, y se necesita un convertidor anal\'ogico digital que funcione a mayor velocidad. Podr\'ia pensarse que cualquier cantidad menor a una infinidad de muestras generar\'a errores en la señal digital y como se ha mencionado anteriormente esto no es posible computacionalmente. Afortunadamente el an\'alisis matem\'atico del proceso de muestreo proporciona las condiciones necesarias para conseguir una representaci\'on digital de la señal an\'aloga sin perdida de informaci\'on, esto quiere decir que la señal puede reconstruirse a partir de las muestras por lo que se considera un proceso reversible, las condiciones bajo las cuales es posible dependen exclusivamente del ancho de banda de frecuencias de la señal y de la frecuencia de muestreo \cite{dodge}.\\ 

Considerando lo anterior y buscando optimizar el proceso de muestreo, en el sentido de conseguir una representaci\'on de la señal sin perdida de informaci\'on sin tener que almacenar o procesar un vol\'umen de informaci\'on excesiva, surge la pregunta: ¿Cu\'al es la frecuencia de muestreo m\'inima para representar correctamente una señal?. El teorema de Nyquist-Shannon nombrado as\'i en honor a Harry Nyquist y Claude Shannon, responde esta pregunta y se enuncia a continuaci\'on.\\

\noindent
\textbf{\textit{Teorema de muestreo.- }}Una señal continua \begin{math}x(t)\end{math} registrada con una frecuencia de muestreo \begin{math}f_s\end{math} de la cu\'al se obtien una copia a tiempo discreto \begin{math}x[n]\end{math}, puede reconstruirse perfectamente a su forma original \begin{math}x(t)\end{math} a partir de la serie \begin{math}x[n]\end{math} si,  \begin{math}f_s > 2f_{max}\end{math}, donde \begin{math}f_{max}\end{math} es la m\'axima frecuencia contenida en el espectro de la señal \begin{math}x(t)\end{math}.\\

A la frecuencia \begin{math}f_N = \frac{1}{2}f_s\end{math} se le denomina frecuencia de Nyquist, o limite de Nyquist, esta frecuencia es el limite superior del ancho de banda del espectro armonico de la señal. Como se mencion\'o anteriormente las frecuencias en la señal continua que sean superiores a la frecuencia de Nyquist, no ser\'an interpretadas correctamente al momento de discretizarse, esto ocurre por un efecto llamado Aliasing que tiene que ser considerado y tratado, previo a la etapa de muestreo \cite{smith}.\\

\noindent
\textbf{\textit{Aliasing}}\\
\indent

El efecto de \textit{aliasing} se produce cuando la señal contiene frecuencias superiores a la frecuencia de Nyquist y no se logra una representaci\'on fiel de dichas frecuencias, esto sucede por que para poder interpolar correctamente una onda se necesita que al menos se tomen dos registros por cada ciclo de onda, en caso de que esto no suceda la frecuencia de la onda ser\'a interpretada como una frecuencia mas baja, esta frecuencia mas baja se llama Alias y depende de la frecuencia original y la frecuencia de muestreo, la frecuencia alias se encontrar\'a debajo de la frecuencia de Nyquist. Si se produce este efecto, la señal discreta tendr\'a un espectro arm\'onico muy distinto al de la señal continua ya que las frecuencias superiores a la frecuencia de Nyquist no s\'olo no estar\'an presentes en la señal discreta, si no que por cada una de estas frecuencias se introducir\'a en la señal discreta su Alias, es decir frecuencias que en la señal original no estaban presentes, por esta raz\'on el efecto de Aliasing es indeseable y debe ser eliminado.\\

En la figura 3.2 se muestra el proceso de muestreo para una señal generada por una funci\'on seno, la frecuencia de la señal es equivalente a 0.91 por ciento de la frecuencia de muestreo, es decir, est\'a por encima de la frecuencia de Nyquist, es f\'acil ver que los valores tomados al muestrear la señal representan una frecuencia mas baja que la frecuencia de la señal. Adicionalmente al cambio en la frecuencia, este efecto tambi\'en puede resultar en un cambio de fase, en la figura se observa un cambio de fase de $180^{\circ}$. Solo es posible que se produzcan dos tipos de cambio de fase $0^{\circ}$ (se conserva la fase) y $180^{\circ}$ (inversi\'on de fase). Para  señales con frecuencia de 0 a 0.5, 1.0 a 1.5, 2.0 a 2.5, etc. en relaci\'on a la frecuencia de muestreo no se produce cambio de fase, para frecuencias de 0.5 a 1.0, 1.5 a 2.0, 2.5 a 3.0 etc. con relaci\'on a la frecuencia de muestreo se produce inversi\'on de fase \cite{smith}.\\

\begin{figure}[H]
	\begin{center}
	\includegraphics[scale=0.4,type=png,ext=.png,read=.png]{imagenes/aliasing} \\
	\caption{Efecto de aliasing producido al registrar una señal con frecuencia mayor a la frecuencia de Nyquist.}
	\label{fig:diag_recon_locutor}
	\end{center}
\end{figure}

La forma en que se elimina este efecto es introduciendo un filtro pasa bajas antes de realizar el muestreo como se muestra en la figura, este filtro tiene como finalidad eliminar las frecuencias superiores a la frecuencia de Nyquist, de esta manera no se introducen frecuencias alias, y se logra una representaci\'on de la señal correcta en el ancho de frecuencias que no son eliminadas por el filtro.\\




\subsubsection{Cuantizaci\'on}

El proceso mediante el cu\'al se convierte una señal de amplitud continua en una de amplitud discreta se llama cuantizaci\'on, este consiste en restringir los valores que pueden ser registrados de tal manera que se asigna un valor cercano al valor real de la amplitud de la señal continua, este proceso a diferencia del de muestreo es no reversible, es decir al realizarlo se pierde informaci\'on. Al n\'umero de posibles valores se le llama profundidad de bits o resoluci\'on, y se expresa en n\'umero de bits, por ejemplo una resoluci\'on de 16 bits ofrece 65,536 posibles valores enteros que van de -32,768 a 32,767 incluyendo al 0, similar al caso del tiempo continuo, a mayor resoluci\'on se necesita mas recursos de procesamiento y almacenamiento, en el caso contrario en que la resoluci\'on sea demasiado baja se produce distorsi\'on de la señal.\\


El conjunto de valores disponibles que puede tomar la señal digital se denomina niveles de cuantizaci\'on, y la distancia entre dos valores sucesivos se conoce como resoluci\'on de cuantizaci\'on o tamaño de paso de cuantizaci\'on y se suela denotar por $\Delta$. Existen varias formas de realizar la cuantizaci\'on, la m\'as com\'un es por truncamiento en donde se asigna a cada muestra el nivel de cuantizaci\'on inmediato inferior al valor real, otra forma mas conveniente es por redondeo, en este m\'etodo se asigna el nivel de cuantizaci\'on mas cercano al valor real \cite{proakis}. \\

Se denota como \begin{math}Q[x(n)]\end{math} a la operaci\'on de cuantizaci\'on sobre el conjunto de muestras \begin{math}x(n)\end{math} y a la secuencia de muestras cuantizadas se denota como \begin{math}x_q(n)\end{math}, por lo que se tiene:
\begin{align}
	x_q(n) = Q[x(n)]
\end{align}

A la diferencia entre la muestra cuantizada y el valor real se le denomina error de cuatizaci\'on o ruido de cuantizaci\'on y se denota por \begin{math}e_q(n)\end{math}, es decir:
\begin{align}
	e_q(n) = x_q(n) - x(n)
\end{align}

Considerando que la cuantizaci\'on se realiza por redondeo, y suponiendo que el valor de la amplitud de la señal se encuentra dentro del rango del cuantizador, se denomina al error resultante, error granular y est\'a acotado de la siguiente forma:
\begin{align}
	-\frac{\Delta}{2} < e_q(n) \leq \frac{\Delta}{2}
\end{align}

Cuando el valor de la señal cae fuera del rango de quantizaci\'on, este tipo de error no est\'a acotado, por lo que puede distorsionar severamente la señal, este error se denomina error de sobrecarga o de clipeo, en general se asume que este tipo de error no estar\'a presente ya que se puede resolver reescalando el rango de la señal para que los valores est\'en dentro del rango de cuantizaci\'on.\\

Lo mas importante a tener en cuenta es el error que se introduce al realizar este proceso, en el modelo de error bajo ciertas condiciones se asume que se distribuye normal con media cero y desviaci\'on standard $\Delta$/$\sqrt{12}$ o aproximadamente 0.29$\Delta$, por ejemplo una resoluci\'on de 8 bits introducir\'a un ruido en rms de 0.29/256 o aproximadamente 1/900, mientras que para una resoluci\'on de 16 bits se introduce un ruido de 0.29/65536 $\approx$ 1/14000 \cite{smith}. Para elegir correctamente la resoluci\'on es necesario conocer la relaci\'on señal/ruido de la señal anal\'oga, ya que si esta es mayor que la relaci\'on señal/rudio de la señal de entrada, el error es no significativo y la señal puede ser reconstruida casi perfectamente despu\'es de la cuantizaci\'on.\\



\subsubsection{Codificaci\'on}
La codificaci\'on es el \'ultimo paso en el proceso de digitalizar una señal an\'aloga, aunque en general se realiza al mismo tiempo que la cuntificaci\'on, en este paso se asigna un n\'umero binario a cada nivel de cuantizaci\'on, esta asignaci\'on debe ser un mapeo biyectivo, es decir a cada nivel se le asigna un \'unico valor binario y dos niveles no pueden compartir el mismo valor, por lo tanto si tenemos L niveles de cuantizaci\'on, se necesitan al menos L distintos n\'umeros binarios para realizar la codificaci\'on \cite{proakis}, es por esto que a la resoluci\'on en general se le define por su n\'umero o profundidad de bits.\\

Aunque existen distintos tipos de esquemas de codificaci\'on y es un proceso importante a tener en cuenta, sobre todo para realizar el procesamiento y al momento de realizar la decodificaci\'on, este no tiene ning\'un efecto en el rendimiento de la cuantificaci\'on o en el procesamiento, por lo que no se realiza mayor menci\'on de la codificaci\'on.\\



\subsection{Transformada de Fourier}

En 1807 el matem\'atico franc\'es Jean Baptist Fourier estudiando la propagaci\'on de calor a trav\'es de ondas sinosiodales para representar la distribuci\'on de temperatura, publica un art\'iculo en donde propone que cualquier funci\'on peri\'odica puede ser representada como la suma de un conjunto de funciones trigonom\'etricas. Esta afirmaci\'on result\'o muy controversial en su \'epoca ya que algunos de sus colegas no estaban convencidos de que se cumpliera en todos los caso y no permitieron que se publicara el art\'iculo, esta controversia estaba bien justificada ya que existen funciones que no pueden ser representadas de manera ex\'acta usando esta t\'ecnica, sin embargo pueden aproximarse a tal grado de que la diferencia sea insignificante, en el caso de señales discretas, la representaci\'on de la señal siempre es exacta \cite{smith}. Con el paso del tiempo ha quedado claro la utilidad de esta teor\'ia y la gran cantidad de aplicaciones.\\

La transformada de Fourier es la representaci\'on de una funci\'on como una suma de funciones trigonom\'etricas de distintas frecuencias, fases y amplitudes, a estas funciones se les llama componentes arm\'onicos o simplemente arm\'onicos. Esta transformaci\'on traduce una funci\'on cuya variable independiente es generalmente el tiempo, en otra donde la variable independiente es la frecuencia, es por esto que se dice que transforma una funci\'on del dominio del tiempo al dominio de las frecuencias. Esta descomposici\'on es muy importante en el an\'alisis funcional, procesamiento de señales digitales y en el an\'alisis de sistemas lineales invariantes en el tiempo ya que bajo estos sistemas una funci\'on sinosoidal siempre conserva su forma y frecuencia y s\'olo modifica su fase y amplitud \cite{proakis}.\\ 

Esta representaci\'on se lleva a cabo encontrando la amplitud y la fase de cada una de las frecuencias que componen a la funci\'on, esto se logra aplicando el producto punto entre la funci\'on y cada uno de los arm\'onicos, de esta forma se logran aislar las frecuencias que componen a una funci\'on, esto funciona por que las funciones trigonom\'etricas de distintas frecuencias son ortogonales, por lo que el producto punto entre distintas frecuencias es cero \cite{ellis}.\\

Existen cuatro tipos de transformada de Fourier que responden a los cuatro tipos b\'asicos de señales \cite{smith}. Una señal puede ser peri\'odica y no peri\'odica, y como se ha mencionado anteriormente puede ser continua o discreta, las combinaci\'ones de estas dos caracter\'isticas resulta en los cuatro tipos b\'asicos de señales y en las distintas versiones de las transformada de Fourier:
\begin{itemize}
	\item[] \textbf{Transformada de Fourier}\\
	Esta versi\'on se usa en caso de tener \textbf{señales continuas aperi\'odicas} y se define como:
	\begin{align}
		X(\omega) =  \int_{-\infty}^{\infty} x(t)e^{-j\omega t} \,dt
	\end{align}
	\setstretch{0.05}
	\begin{align*}
		\omega \in (-\infty,\infty)
	\end{align*}
	\setstretch{1.25}
	\item[] \textbf{Serie de Fourier}\\
	Esta versi\'on se usa en caso de tener \textbf{señales continuas peri\'odicas} y se define como:
	\begin{align}
		X(k) =  \int_{0}^{P} x(t)e^{-j\omega_{k} t} \,dt
	\end{align}
	\setstretch{0.05}
	\begin{align*}
		k = -\infty, ..., \infty
	\end{align*}
	\setstretch{1.25}
	\item[] \textbf{Transformada de Fourier a Tiempo Discreto}\\
	Esta versi\'on se usa en caso de tener \textbf{Señales discretas aperi\'odicas}  y se define como
	\begin{align}
		X(\omega) = \sum_{n=-\infty}^{\infty} x(n)e^{-j\omega n}
	\end{align}
	\setstretch{0.05}
	\begin{align*}
		\omega \in (-\pi,\pi)
	\end{align*}
	\setstretch{1.25}
	\item[] \textbf{Transformada Discreta de Fourier}\\
	Esta versi\'on se usa en caso de tener \textbf{Señales discretas peri\'odicas}  y se define como\\
	\begin{align}
		X(k) = \sum_{n=0}^{N - 1} x(n)e^{-j\omega_{k} n}
	\end{align}
	\setstretch{0.05}
	\begin{align*}
		k = 0, ..., N - 1
	\end{align*}
	\setstretch{1.25}
	\setlength{\parskip}{2em}
\end{itemize}

Con estos cuatro casos es posible encontrar la representaci\'on espectral de cualquier funci\'on ya que siempre va a caer en alguno de estos, en el contexto de este trabajo s\'olo es relevante la transformada discreta de Fourier, ya que este caso es el que se aplica en señales discretas y finitas que son el tipo de señales con que se trabaja en el procesamiento digital.\\

%Para ganar un poco de intuici\'on sobre las f\'ormulas anteriores, es claro que para señales continuas se tiene que integrar, para señales aperiodicas el   tomemos una señal peri\'odica, las frecuencias que componen esta señal son m\'ultiplos de la frecuencia fundamental de la señal, esto es as\'i por que solo estas frecuencias completaran un n\'umero exacto de ciclos dentro del periodo de la señal, .\\

\noindent
\textbf{Transformada inversa}\\
\indent
Expresar una funci\'on en t\'erminos de sus frecuencias se llama an\'alisis de Fourier y por ser una representaci\'on alterna de la funci\'on es un proceso reversible, es decir a partir de una serie de Fourier que se encuentra en el dominio de las frecuencias, siempre se puede pasar al dominio del tiempo expl\'icitamente, esto se realiza con la transformada inversa de Fourier mediante un proceso llamado s\'intesis de Fourier, en donde simplemente se suma el valor de cada uno de los arm\'onicos en cada instante de tiempo. Ambas representaciones la del dominio del tiempo y el dominio de las frecuencias se denominan el par de transformaci\'on y son representaciones equivalentes de la misma funci\'on \cite{ellis} 


\subsubsection{Propiedades}
La transformada de Fourier y la transformada inversa de Fourier es la relaci\'on que guarda una señal en sus dos representaciones, el dominio del tiempo y el de las frecuencias, y cualquier cambio que sufra la señal en alguno de sus dominios se ver\'a reflejado en el otro, es decir, cualquier operaci\'on que se realice a la señal en alguno de sus dominios, se ver\'a reflejada de alguna forma en el dominio opuesto, la relaci\'on en que un cambio matem\'atico en un dominio produce un cambio en el dominio opuesto son las propiedades de la transformada de Fourier, estas propiedades resultan muy \'utiles ya que alguna operaci\'on puede resultar m\'as sencilla de realizar si se cambia de dominio, si este es el caso se suele cambiar de dominio aplicar la operaci\'on y volver a cambiar de dominio. A continuaci\'on se enlistan algunas de las propiedades:\\ 

Sean f(t) y g(t) tales que: f(t) $\longleftrightarrow$ F($\omega$) y g(t) $\longleftrightarrow$ G($\omega$), entonces se cumplen las siguientes propiedades:\\

\noindent
\textbf{\textit{Linealidad}}\\
\indent
\setstretch{0.05}
\begin{align*}
	af(t) + bg(t) \longleftrightarrow aF(\omega) + bG(\omega)
\end{align*} 
\setstretch{1.25}

\noindent
\textbf{\textit{Translaci\'on}}\\
\indent
\setstretch{0.05}
\begin{align*}
	 f(t-t_{0}) \longleftrightarrow e^{-j\omega t_{0}}F(\omega)\\
\end{align*} 
\setstretch{1.25}

\noindent
\textbf{\textit{Cambio de escala}}\\
\indent
\setstretch{0.05}
\begin{align*}
	f(at) \longleftrightarrow \frac{1}{\left| a \right|}F(\frac{\omega}{a})\\
\end{align*} 
\setstretch{1.25}

\noindent
\textbf{\textit{Convoluci\'on}}\\
\indent
\setstretch{0.05}
\begin{align*}
	 f(t)*g(t) \longleftrightarrow F(\omega)G(\omega)\\
\end{align*} 
\setstretch{1.25}

\noindent
\textbf{\textit{Modulaci\'on}}\\
\indent
\setstretch{0.05}
\begin{align*}
	 f(t)g(t) \longleftrightarrow \frac{1}{2\pi}[F(\omega)*G(\omega)]\\
\end{align*} 
\setstretch{1.25}


\subsubsection{Transformada discreta de Fourier}

La transformada de Fourier tiene la particularidad de que asume que el dominio de la funci\'on se extiende de menos infinito a infinito, esto puede parecer una contrariedad ya que en el contexto del procesamiento de señales digitales, se emplean señales con un n\'umero finito de muestras. Hay dos formas de afrontar este supuesto y hacer que la señal finita parezca una señal infinita suponiendo que hay una infinidad de muestras a la izquierda y a la derecha de la señal, en el primer caso es suponer que estas muestras tiene un valor de cero, en este caso la señal se considera discreta y a peri\'odica, el segundo caso es haciendo que la señal sea discreta y peri\'odica, suponiendo que las muestras añadidas son copias exactas y desplazadas de la señal finita.\\

En el caso en que se construye una señal aperi\'odica se necesita una infinidad de ondas sinosoidales para su representaci\'on, lo cu\'al es imposible para un algoritmo computacional, mientras que en el caso de una señal peri\'odica si es posible representarla con un conjunto finito de ondas sinosoidales, este es la \'unica forma posible de implementar una transformada de Fourier computacionalmente ya que las computadoras solo pueden procesar informaci\'on finita y discreta.\\ 

Es una de las herramientas mas importantes en el procesamiento de señales digitales, dentro de sus usos m\'as comunes esta, calcular el espectro de frecuencias de una señal, encontrar la respuesta de frecuencias de un sistema a trav\'es de la respuesta de impulso o viceversa, esto permite analizar los sistemas desde el dominio de las frecuencias, tambi\'en puede ser empleado como paso intermedio en procesos mas complejos, como la convoluci\'on de la Transformada r\'apida de Fourier un algoritmo eficiente para convolucionar señales.\\

La transformada discreta de Fourier convierte una señal \textbf{x[ ]} en el dominio del tiempo de N muestras en una señal \textbf{X[ ]} en el dominio de las frecuencias compuesta por dos series de N/2 + 1 valores cada una, estas dos series corresponden a los valores de la amplitud de la funci\'on coseno y la funci\'on seno respectivamente, la funci\'on coseno es la parte Real de \textbf{X[ ]} y se denota por \textbf{ReX[ ]}, mientras que la funci\'on seno es la parte imaginaria de \textbf{X[ ]} y se denota como \textbf{ImX[ ]}, la suma de la parte real con la imaginaria nos da un n\'umero complejo que lleva la informaci\'on de amplitud y fase para cada arm\'onico.\\

Puede ser calculada de varias formas obteniendo el mismo resultado en todos los casos, el problema se puede abordar como un conjunto de ecuaciones simultaneas, tambi\'en se puede resolver usando la correlaci\'on entre la señal y las señales trigonom\'etricas, sin embargo de entre las posibles formas de calcularla, existe un m\'etodo que es cientos de veces m\'as vel\'oz que los dem\'as y en la mayoria de los casos se prefiere por su eficiencia computacional, este m\'etodo se presenta a continucaci\'on \cite{smith}.\\

\subsubsection{Transformada r\'apida de Fourier}

Aqui falta.\\

Existen varias formas de calcular la transformada discreta de Fourier y todas producir\'an el mismo resultado, sin embargo existe un m\'etodo que es computacionalmente mas eficiente que cualquier otro, reduciendo el tiempo de procesamiento considerablemente, este algoritmo se llama transformada r\'apida de Fourier y se atribuye su descubrimiento a J.W. Cooley y a J.W.Tukey.\\

\subsection{Espectrograma}
Graficar una señal en el dominio del tiempo no resulta muy \'util ya que no hay mucha informaci\'on que podamos extraer visualmente, en cambio en el dominio de las frecuencias si que resulta \'util, el espectrograma es una representaci\'on visual en tres dimensiones de la din\'amica de frecuencias de una señal, con esta herramienta se puede observar la evoluci\'on del contenido espectral de una señal en el tiempo, en el contexto del procesamiento de voz tambi\'en se le conoce como huella de voz (voice print).\\

Para generar un espectrograma se usa un mapa de calor, en el eje x generalmente se asigna el tiempo y en el eje y las frecuencias, aunque en algunos casos estos pueden estar invertidos. El rango de frecuencias va de 0 Hz correspondiente al nivel de corriente directa de la señal, a la frecuencia de Nyquist esta es la frecuencia mas alta de la señal posterior al muestreo \cite{beigi2011}. El brillo o el color en cada coordenada representa la cantidad de energ\'ia registrada en un intervalo de frecuencias en un cierto intervalo de tiempo. La señal es dividida en bloques de un n\'umero constante de muestras, la cantidad generalmente es alguna potencia de 2 ya que de esta forma el algoritmo es mas eficiente, a cada segmento se aplica una funci\'on ventana y se obtiene el espectro arm\'onico con la transformada r\'apida de Fourier o con un banco de filtros.\\ 

\begin{figure}[H]
	\begin{center}
	\includegraphics[scale=0.5,type=png,ext=.png,read=.png]{imagenes/espectrograma} \\
	\caption{Resoluci\'on del espectrograma de frecuencias.}
	\label{fig:diag_recon_locutor}
	\end{center}
\end{figure}

La resoluci\'on del espectrograma depende de la frecuencia de muestreo la cu\'al determina la resoluci\'on de frecuencias y del n\'umero de muestras del bloque el cu\'al determina la resoluci\'on en el tiempo. Para ejemplificar esto, en la figura 3.2 se muestra el espectrograma de frecuencias de la misma señal en cuatro distintas resoluciones. Las figuras de la parte superior tienen una frecuencia de muestreo de 12KHz lo que permite tener una resoluci\'on de frecuencias de 0 a 6KHz, para las figuras de la parte inferior la frecuencia de muestreo es de 4KHz, lo que permite una resoluci\'on de frecuencias de 0Hz a 2KHz. Las figuras de la derecha tienen un tamaño de bloque de 216 muestras y las del lado derecho de 512 muestras, se puede apreciar que la resoluci\'on temporal es mejor cuando el n\'umero de muestras por bloque es menor.\\


\section{Procesamiento de la señal de voz}
\subsection{Detecci\'on de actividad de voz (VAD)}
El procesamiento de la señal de voz incluye procesos que pueden ser muy demandantes computacionalmente, sin embargo en la pr\'actica la señal de voz es discontinua ya que presenta silencios entre frases, incluso en una misma frase se pueden presentar pequeños momentos de silencio, cerca de un 30$\%$ de las muestras en una grabaci\'on de audio normal son silencio, esto quiere decir que removiendo el silencio de la señal se puede lograr un aumento de la velocidad de los procesos en un porcentaje similar \cite{beigi2011}. Una forma de ahorrar recursos computacionales, es detectando estos fragmentos donde no hay señal de voz y ya se eliminarlos o ignorarlos para que no se realicen los procesos mas demandantes.\\

Detecci\'on de actividad de voz hace referencia a la tarea de identificar si en una señal hay presencia de voz o no, as\'i que puede ser traducido en un problema de clasificaci\'on binaria, los segmentos en los que no hay presencia de actividad de voz pueden ser silencio, pero tambi\'en se puede incluir ruidos de ambiente como sonidos de puertas, ventanas, autos o incluso sonidos de respiraciones, carraspeo, tos, etc \cite{gold}. Es importante tomar en cuenta la naturaleza de estos segmentos ya que el tratamiento de fragmentos silencio puede ser significativamente mas sencillo que el tratamiento de otra clase de sonidos.\\

Si consideramos que los fragmentos donde no hay presencia de voz son de silencio o de ruido de ambiente, el enfoque m\'as eficiente dada su sencillez y eficacia es el de definir un umbral para el nivel de energ\'ia, se asume que para los segmentos en donde no hay presencia de voz la energ\'ia de la señal aumenta considerablemente en comparaci\'on con los fragmentos de silencio, esta caracter\'istica es explotada midiendo la energ\'ia de la señal y si esta cae por debajo del umbral el fragmento analizado se considera como silencio \cite{beigi2011}.\\

\begin{figure}[H]
	\begin{center}
	\includegraphics[scale=0.45,type=png,ext=.png,read=.png]{imagenes/vad_db} \\
	\caption{Detecci\'on de actividad de voz con umbral de energ\'ia de -11 db.}
	\label{fig:diag_recon_locutor}
	\end{center}
\end{figure}

Para implementar el enfoque de umbral de energ\'ia primero es necesario calcular la energ\'ia de la señal (f\'ormula 3.9), este c\'alculo se realiza en ventanas de aproximadamente 30 ms de duraci\'on con un traslape de 50 \%, la energ\'ia de la señal se suele expresar en decibeles mediante la f\'ormula $10\log_{10}(\sigma^2(x))$, posteriormente es recomendado realizar un an\'alisis de la señal para estimar cu\'al podr\'ia ser un buen umbral, una vez definido el umbral se clasifica la señal en segmentos con actividad de voz y en segmentos sin actividad de voz (f\'ormula 3.8). En la figura 3.5 se muestra un ejemplo de detecci\'on de actividad de voz con un umbral de energ\'ia de $\theta_{silencio}$ := -11 db.\\

\begin{align}  
VAD(x) := 
	\begin{cases}
		0, &\quad\text{if } \sigma^2(x) < \theta_{silencio}\\
		1, &\quad\text{if } \sigma^2(x) \ge \theta_{silencio}\\
	\end{cases}
\end{align}\\

donde:\\

\begin{align}
   \sigma^2(x) &:= \lVert x \rVert^2  =  \sum_{k=0}^n x_k^2
\end{align}\\

Existen otros m\'etodos mas sofisticados para identificar segmentos con presencia de actividad de voz, como la correlaci\'on en la señal, m\'axima verosimilitud, redes neuronales y en general las t\'ecnicas utilizadas para el reconocimiento de voz y de locutor tambi\'en pueden ser empleadas para la detecci\'on de actividad de voz \cite{beigi2011}, estas t\'ecnicas no se presenta, ya que en este trabajo la t\'ecnica de umbral de energ\'ia es suficiente.\\

\subsection{Escalas de frecuencias psicoac\'usticas}

Aqui falta.\\

\subsubsection{Percepci\'on auditiva}

Aqui falta.\\

\subsubsection{Escala Mel}

Aqui falta.\\

La escala de frecuencias Mel es una relacion perceptual de la frecuencia obtenida experimentalmente, en el que el observador juzga que los elementos estan separados a distancias iguales. Dado que la forma en que percibimos la frecuencia no es lineal, en la decada de 1930 se hizo un estudio con la finalidad de modelar la relación perceptual de la frecuencia, cinco sujetos fraccionaron 10 tonos a distintas frecuencias, y los valores obtenidos se utilizaron para construir una escala num\'erica proporcional a la magnitud percibida del tono. En dicho estudio por ser de caracter experimental los resultados se reportaron en una gr\'afica que relaciona la frecuencia con el tono. De dicho estudio se desarrollo la siguiente definición: 

Definición Mel. Mel que es una abreviacion de la palabra melody. Es una unidad de Tono. Se define como una cent\'esima de tono de una onda simple con frecuencia de 1000 Hz y con una amplitud de -40dB.

También de dicho estudio surgen las siguientes dos fórmulas: 

\begin{align}
  F_{Mels} &= \frac{1000}{\ln(1 + \frac{1000}{700})} \cdot  \ln{\left(1 + \frac{F_{Hz}}{700}\right)}\
\end{align}

\begin{align}
  F_{Mels} &= \frac{1000}{\log(2)} \cdot \log{\left(1 + \frac{F_{Hz}}{1000}\right)}\
\end{align}

Donde F es la frecuencia en Herz y F la frecuencia en Mels. La primera gracias a O'Shaughnessy y la segunda atribuida a Fant, quienes de manera independiente ajustaron una fórmula a los datos reportados por Stevens y Volkmann.


Es el resultado de un estudio realizado por Stevens, Volkman, Newman en 1930, y publicado en 1937 en donde se demostro experimentalmente que la relación entre frecuencia y tono no es lineal, observaron que para frecuencias abajo de los 1000 hz la relación puede considerarse como lineal y para las frecuencias superiores a los 1000 hz  puede considerarse como logaritmica. 


Pitch.- Es una cantidad percibida relacionada a la frecuencia fundamental de vibración en un determinado lapso de tiempo, la unidad de medida del Pitch es el Mel.

\subsubsection{Escala Bark}

Aqui falta.\\

\subsection{M\'etodos de extracci\'on de caracter\'isticas de la señal de voz}

La extracci\'on de caracter\'isticas hace referencia al computo de un conjunto de variables llamadas vectores de caracter\'isticas que se calculan a partir de una señal de voz representada como serie de tiempo, de esta forma se transforma la señal de voz observada en una representaci\'on param\'etrica sobre la cu\'al se realiza el an\'alisis y procesamiento. La finalidad de la extracci\'on de caracter\'isticas es encontrar una transformaci\'on a un espacio de dimensi\'on menor que preserve la informaci\'on necesaria para poder realizar comparaciones usando medidas de similitud \cite{campbell1997}, el vector de caracter\'isticas constituye una representaci\'on comp\'acta de la señal de voz que permite generar y almacenar los modelos de locutores de forma eficiente, al mismo tiempo que reduce la dimensionalidad de los datos conservado la informaci\'on necesaria que le permita al sistema diferenciar efectivamente a los locutores.\\

Las caracter\'isticas deseadas en un sistema de reconocimiento de locutor, deben poseer los siguientes atributos:
\begin{itemize}
	\item F\'aciles de extraer y de medir.
	\item No deben verse afectadas por el estado f\'isico del locutor.
	\item Deben ser consistentes en el tiempo.
	\item No deben verse afectadas por el ruido ambiente.
	\item Que no sean imitables.\\
\end{itemize}

Estos atributos son deseables, sin embargo ninguna parametrizaci\'on de la señal de voz posee todos los atributos mencionados anteriormente, se ha demostrado que las caracter\'isticas basadas en el an\'alisis espectral son las m\'as efectivas en los sistemas de reconocimiento autom\'atico \cite{feng}. En el contexto de los sistemas de reconocimiento de voz y de reconocimiento de locutor, las caracter\'isticas de la señal de voz basadas en coeficientes cepstrales o escalas perceptuales son ampliamente usadas, m\'etodos como el de coeficientes cepstrales de frecuencias Mel (MFCC), codificaci\'on predictiva lineal (LPC) y predicci\'on lineal perceptiva (PLP) son de los m\'as usados en procesamiento de señales de voz \cite{namrata}.\\


\subsubsection{Coeficientes cepstrales de frecuencias Mel (MFCC)}

El m\'etodo mas predominante para extraer caracter\'isticas espectrales de la señal de voz es sin lugar a dudas el m\'etodo de coeficientes cepstrales de frecuencias Mel, esta es una t\'ecnica basada en el an\'alisis espectral usando la escala de Mel, por lo que se considera un m\'etodo que trata de emular la percepci\'on auditiva humana. Los MFCC's son una parametrizaci\'on robusta ante variaciones en el habla del locutor y variaciones en las condiciones de grabaci\'on, se emplean t\'ecnicas de extracci\'on de informaci\'on similares a las usadas por los humanos, al mismo tiempo que se desenfat\'iza otra informac\'on considerada no relevante o que pueda obstruir la caracterizaci\'on de la voz \cite{namrata}.\\

Una parte importante de la señal de voz es su evoluci\'on en el tiempo, para capturar esta din\'amica se incluye en el vector de caracter\'isticas las diferencias de primer y segundo orden de los MFCC's, en general las diferencia de primer y segundo orden son independientes de los MFCC's y se incluyen para modelar la din\'amica local de la señal de voz, estas diferencias son conocidas como \textit{Coeficientes Cepstrales Deltas} y \textit{Coeficientes Cepstrales Delta-Deltas} \cite{beigi2011}.\\

Com\'unmente los coeficientes cepstrales de frecuencias Mel son vectores de dimensi\'on 39, ya que consisten de 13 coeficientes est\'aticos, 13 coeficientes para las diferencias de primer orden y 13 coeficientes para las diferencias de segundo orden, aunque en \cite{beigi2011} se recomienda que para las diferencias de primer orden se use un n\'umero menor de coeficientes y un n\'umero a\'un menor para los coeficientes de las diferencias de segundo orden considerando la cantidad de datos disponibles en las etapas de entrenamiento y prueba. A continuaci\'on se detallan los pasos que componen su c\'alculo:\\

Poner un diagrama de los pasos\\

\noindent
\textbf{\textit{Pre-\'enfasis}}\\
\indent
Dadas las caracter\'isticas del sistema vocal humano, los componentes vocales en las frecuencias altas presentan reducci\'on de energ\'ia, como resultado, la energ\'ia se reduce de manera lineal, y aunque para cada locutor esta reducci\'on puede variar dependiendo de varios factores, se puede afirmar que en promedio la energ\'ia se reduce a un ritmo de 2db/kHz. Esta reducci\'on de energ\'ia genera problemas de implementaci\'on, por ejemplo si se aplica la transformada discreta de Fourier usando aritm\'etica de punto fijo la precisi\'on ser\'a muy distinta en distintas partes del espectro arm\'onco, presentando una precisi\'on menor para frecuencias altas.\\

Pre-\'enfasasis es una t\'ecnica usada comunmente en el procesamiento de la señal de voz, sirve para centrar la señal al rededor del cero, aplanar el espectro arm\'onico enfatizando las frecuencias altas, y produce una señal menos susceptible a errores de precisi\'on en pasos posteriores en el procesamiento de la señal. Generalmente el pre-\'enfasis se realiza en el dominio del tiempo aplicando un filtro FIR de primer orden a la señal digitalizada x(t) de la siguiente manera:
\begin{align}
   y(t) &= x(t) - \alpha \cdot x(t - 1)
\end{align}

Donde el coeficiente $\alpha$ es el factor de pre-\'enfasis, este puede tomar valores entre 0.95 y 0.99, en \cite{feng} se recomienda utilizar un factor de 0.97. Hay que considerar que aplicar pre-\'enfasis excesivamente puede ocasionar problemas en consonantes fricativas, ya que estas tienen mayor energ\'ia en las frecuencias altas, y tambi\'en produce una modificaci\'on tanto perceptiva como en el modelado estad\'istico, as\'i que el factor de pre\'enfasis depender\'a de las necesidades de la aplicaci\'on e implementaci\'on del sistema.\\

\noindent
\textbf{\textit{Segmentaci\'on}}\\
\indent
La variedad de fonemas que contiene cualquier lenguaje es tan basta que hace que la señal de voz sea muy compleja, en general un fragmento de discurso hablado es mucho mas complicado que simplemente una consonante fricativa o una vocal sostenida, es por esto que la transformada de Fourier que se aplica com\'unmente en señales peri\'odicas o estacionarias no es aplicable para señales de voz cuyas propiedades cambian notoriamente a trav\'es del tiempo. Sin embargo se ha observado que propiedades temporales como la energ\'ia, los cruces por el cero y la correlaci\'on se pueden asumir fijas en intervalos de tiempo de entre 10 a 30 mili segundos, lo mismo pasa en el caso de las caracter\'isticas espectrales, se puede asumir que en estos lapsos cortos de tiempo el contenido espectral cambia relativamente lento y la señal puede considerarse estacionaria \cite{rabiner1987}.\\

Por esta raz\'on para realizar el an\'alisis de la señal con la finalidad de obtener una buena aproximaci\'on del contenido espectral, se suele dividir en pequeños fragmentos que como se ha mencionado anteriormente suelen tener una duraci\'on entre 0.01 y 0.03 segundos, la señal es una lista de muestras indexadas, por lo tanto para convertir la duraci\'on de la señal de segundos a n\'umero de muestras se utiliza la siguiente f\'ormula:
\begin{align}
   N_{muestras} = \lceil segundos*f_{muestreo} \rceil
\end{align}

Es decir, el n\'umero de muestras de cada fragmento es igual al menor entero mayor o igual que los segundos multiplicados por la frecuencia de muestreo. Para cubrir la longitud total de la señal estos fragmentos se van desplazando de tal forma que haya una superposici\'on entre fragmentos vecinos, es decir el n\'umero de muestras del desplazamiento es menor que el n\'umero de muestras de cada fragmento, el desplazamiento com\'unmente varia entre el 30 y 50 \% de la duraci\'on total del fragmento. Una vez realizado este proceso, se pasa de una lista de muestras a una matriz donde el n\'umero de filas es el n\'umero de segmentos en que se dividi\'o la señal y el n\'umero de columnas es el n\'umero de muestras de cada segmento, en caso de que el \'ultimo segmento tenga menos muestras que el resto, se suele asignar al n\'umero de muestras faltantes el valor cero.\\


Es importante elegir correctamente el tamaño de los fragmentos y el tamaño del desplazamiento, la duraci\'on promedio de un fonema es de 80 ms  por lo que podr\'ia ser un buen candidato para el tamaño de los fragmentos, sin embargo es un promedio sesgado ya que las vocales son mucho mas largas que algunas consonantes que pueden tener una longitud de 5 ms, por lo que se considera que una longitud de fragmento de entre 20 ms y 30 ms con un desplazamiento de 10 ms logra captar correctamente tanto vocales como consonantes, siendo estos valores los m\'as comunes en la pr\'actica \cite{beigi2011}.\\

\noindent
\textbf{\textit{Windowing}}\\
\indent
Una vez que se ha segmentado la señal, se tiene que transformar cada segmento del dominio del tiempo al dominio de las frecuencias para extraer su contenido arm\'onico, para esto es necesario aplicar la transformada discreta de Fourier. Como se ha mencionado antes, al aplicar la transformada discreta de Fourier se asume que la señal es infinita y peri\'odica y que el inicio de la señal esta unido al final de señal, en general, el valore de la primer muestra ser\'a distinto del valor de la muestra final, esto causa que al unirlos haya una discontinuidad de salto, esto generar\'a una distorsi\'on del espectro arm\'onico denominada mancha espectral o leakage, introduciendo frecuencias que no estaban presentes en la señal original.\\

Para no tener este problema, se tiene que suavizar la uni\'on de las muestras inicial y final, una forma de lograr esto es aplicando una func\'on window para atenuar tanto el inicio y el final de la señal. Algunas de las funciones window mas comunes son Hamming, Hann, Triangular, Gauss, Blackman y Bartlett \cite{beigi2011}. Cada una de estas funciones tiene caracter\'isticas distintas, siendo de las mas relevantes, la forma, que tan r\'apido decaen los valores en los extremos, y los valores que toman en los extremos, ya que algunas de las funciones toman el valor cero, mientras otras a pesar de tomar valores muy pequeños, son valores distintos de cero. A continuaci\'on se mencionan algunas funciones window:\\

\noindent
\textbf{\textit{Hamming Window}}\\
\indent
La ventana Hamming es la funci\'on ventana mas usada en el procesamiento de voz, su ecuaci\'on esta descrita por (3.14), una de las razones por la cu\'al es tan popular es por que a pesar de que su espectro decae r\'apidamente, los arm\'onicos de frecuencias altas permanecen planos y se logra cubrir la mayor parte del espectro de una forma homog\'enea.
\begin{align}
   w(n) = 0.54 - 0.46 \cos(\frac{2\pi n}{N - 1})
\end{align}

\noindent
\textbf{\textit{Hann Window}}\\
\indent
La ventana Hann es una variaci\'on de la ventana Hamming, la f\'ormula est\'a dada por (3.15), la mayor diferencia es que la ventana Hann toma el valor cero en sus extremos, lo que puede ser deseable en ciertas circunstancias, aunque un argumento en contra de esta caracter\'istica es que el fragmento de señal no es usado en su totalidad, en el procesamiento de voz no es problema debido a la sobreposici\'on de los fragmentos. 
\begin{align}
   w(n) = 0.5(1 - \cos(\frac{2\pi n }{N - 1})) 
\end{align}

\noindent
\textbf{\textit{Blackman Window}}\\
\indent
La f\'ormula en (3.16) describe una familia de funci\'ones ventana llamada Blackman-Harris, cuando $\alpha = 0.16$ la funci\'on resultante se conoce como la ventana cl\'asica Blackman o simplemente la ventana Blackman. Esta funci\'on ventana conjunta lo mejor de la ventana Hamming y de la ventana Hann ya que posee un espectro que decae r\'apidamente, con un l\'obulo principal grande y l\'obulos secundarios estrechos. Estas prpiedades hacen de la ventana Blackmann una buena opci\'on para el procesamiento de voz.\\
\begin{align}
   w(n) = a_0 - a_1\cos(\frac{2\pi n}{N - 1}) + a_2\cos(\frac{4\pi n}{N - 1})
\end{align}
donde:\\
\indent
$a_0 = \frac{1 - \alpha}{2}$\\
\indent
$a_1 = \frac{1}{2}$\\
\indent
$a_2 = \frac{\alpha}{2}$\\

\noindent
\textbf{\textit{Power spectrum}}\\
\indent

\noindent
\textbf{\textit{Filter bank}}\\
\indent

\noindent
\textbf{\textit{Mean Normalization}}\\
\indent

\noindent
\textbf{\textit{Deltas y Delta-Deltas}}\\
\indent

Como su nombre lo indica el Mel filter bank es un banco de filtros, estos filtros son triángulos idénticos y se encuentran equidistantes sobre la escala de Mel, cubriendo el espectro desde 0 Hz a la mitad de la frecuencia de sampleo, el número de filtros va de 20 a 40 dependiendo de la frecuencia de sampleo, despues estos filtros se pasan de la escala de Mel a Herz. Se convierte a Herz modificando la relación entre los filtros por que despues se aplicara al espectro armónico de una señal. Este banco de filtros busca imitar la estructura de las bandas criticas del oido humano. Genera el efecto de dbrindar mayor resolución a frecuencias bajas\\

El primer paso es encontrar el mí­nimo y máximo de en la escala Mel, el mí­nimo es 0 y el máximo es la mitad de la frecuencia de sampleo en herz convertida en Mels. i.e.
\begin{align}
  Min &= 0\
\end{align}
\begin{align}
  Max &= \frac{1000}{\ln(1 + \frac{1000}{700})} \cdot  \ln{\left(1 + \frac{\frac{SampleRate}{2}}{700}\right)}\
\end{align}

Ahora dentro de este intervalo generamos n + 2 puntos equidistantes donde n es el número de filtros:

\begin{align}
  m_{i} &= Min + i \cdot \frac{Max - Min}{n + 1}  \qquad  i = 0, ..., n + 1
\end{align}

Despu\'es convertimos a Herz

\begin{align}
   h_{i} &= \frac{1000}{\ln(1 + \frac{1000}{700})} \cdot  \ln{\left(1 + \frac{m_{i}}{700}\right)}\ \qquad  i = 0, ..., n + 1
\end{align}



\textbf{Filter bank}
Ahora a la potencia del espectro se aplica el banco de filtro Mel, generalmente 40 filtros triangulares, como habiamos mencionado esto es para obtener caracteristicas perceptuales en la escala de audición humana

\begin{align}
   E^{X}_{i} = \sum_{k = 1}^{N} \mid X(k) \mid ^2  \cdot \psi_{i}(k)
\end{align}

Donde X(k) es la amplitud del espectro, k es el í­ndice de la frecuencia, psi es el i-esimo filtro Mel, E es la energia del banco de filtro

\textbf{Logaritmo natural y la transformada de coseno discreta}
Aplicar el logaritmo natural aproxima la relación entre intensidad del sonido y el volúmen, y tambien convierte la multiplicación de parámetros en una suma, haciendo mas eficiente su cálculo. Dado que estos coeficientes estan sumamente correlacionados y eso puede ser un inconveniente para algunos algoritmos de aprendizaje de máquina, se aplica la transformada de coseno discreta para eliminar esta correlación. 

\begin{align}
   C^{X}_{n} = \sum_{i = 1}^{M} \log(E^{X}_{i}) \cdot \cos[n \frac{(2i - 1)\pi}{2M}]
\end{align}

Finalmente, para la mayoria de las aplicaciones de reconocimiento de voz de los coeficientes cepstrales obtenidos, se toman los coeficientes de menor orden generalmente entre 13 y 20 y se deshecha el resto, ya que estos coeficientes representan la forma del tracto vocal, esto conduce a una reducción dimensional.

\subsubsection{Codificaci\'on lineal predictiva (LPC)}

Una de las t\'ecnicas mas poderosas y predominantes en el procesamiento de voz es el an\'alisis lineal predictivo, su importancia radica en su capacidad de estimar de manera precisa par\'ametros de voz como el tono, formantes, espectro, y las funciones del tracto vocal, as\'i como por su velocidad de computo. Esta t\'ecnica tambi\'en destaca como un m\'etodo de compresi\'on de la señal de voz, ya que puede reducir la cantidad de datos necesarios para su representaci\'on, lo que imp\'acta de forma positiva en procesos de transmisi\'on y almacenamiento. \\

Speech model.\\
Esta t\'ecnica esta relacionada con el modelo de producci\'on y s\'intesis de voz

\cite{ney}
\cite{gold}
\cite{campbell1997}
\cite{rabiner1987}
\cite{beigi2011}\\


\subsubsection{Predicci\'on lineal perceptiva (PLP)}





% Marco Teorico.
\chapter{Clasificadores y metodolog\'ia de evaluaci\'on} \label{chap:baafda}


\section{Gaussian Mixture Models}
\section{Vector Quantization}
\section{Evaluaci\'on}
\subsection{Accuracy}	
\subsection{Precision}
\subsection{Recall}
\subsection{F1 Score}	

% Marco Teorico.
\chapter{Iadad} \label{chap:baafda}




\chapter{Metodología}\label{chapter:metodologia}

% AQU\'I VA EL DESARROLLO DEL PLANTEAMIENTO DEL PROBLEMA.
%
% Los requerimientos de la herramienta o software a implementar se mencionan a continuaci'on: 


\section{Elección de la base de datos} \label{sect:eleccion_bd}
%Se busca para este proyecto una base de datos que recree un ambiente en el cual 
%La selección de la base de datos para este proyecto tiene gran importancia, ya que un modelo entrenado con

\chapter{Evaluación y resultados}\label{chapter:resultados}




\section{Resumen de los resultados} \label{sect:resume_resultado}

Muy recomendable escribir un resumen de resultados.

\chapter{Conclusiones y Trabajo Futuro} \label{chap:conclusiones}


\section{Conclusiones} \label{sect:conclusiones}

De este trabajo se concluye que:

\begin{itemize}
\item{dfdas}


\end{itemize}


\section{Trabajo futuro} \label{sect:trabajo_futuro}

Este trabajo presenta varias áreas en las cuales puede ser mejorado y continuado. A continuación se habla de algunas de ellas:

\begin{itemize}
\item{xfaefa}

\end{itemize}













% Establece las citas y bibliografia
%\bibliographystyle{alpha.bst}
\bibliographystyle{apalike}
\bibliography{myrefs}

% Crea el apendice
\appendix
\chapter{Resultados adicionales}\label{chapter:apendice}


%\chapter{Gram'atica del Lenguaje}

%---------------------------------------------------------------------------------------%
\section {Representación de Dominio}

El archivo por ejemplo para el nodo ...


%\chapter*{Glosario}

\pdfbookmark[0]{Glosario}{glosario}

\subsection*{Aleatorio}
Indica la condición de que una acción produzca resultados con un cierto grado de azar. En la computación el concepto ``aleatorio'' es utópico y por ende, en su lugar se utiliza la palabra ``pseudoaleatorio'' que indica que el resultado no es totalemnte al azar. 7

\subsection*{Grafo}
Es una estructura de datos que representa la forma en la que muchos entes denominados ``nodos'' estan relacionados entre si por valores o condiciones específicas llamados ``arcos''. 9


\end{document}
